% !TeX root = ../main.tex
\documentclass[./../main.tex]{subfiles}

\begin{document}
\subsection{Mô tả vấn đề}

Thế kỉ XXI, thế kỉ của sự hội nhập và toàn cầu hoá, đã và đang có rất nhiều nền văn hóa du nhập và được đón nhận một cách mạnh mẽ tại Việt Nam, đặc biệt phải kể đến là văn hóa Nhật Bản. Một trong các loại hình văn hóa Nhật Bản vô cùng được ưa chuộng tại Việt Nam chính là văn hóa truyện tranh (manga). Cùng với nhu cầu đọc truyện ngày càng lớn của độc giả nước nhà, ngày càng nhiều các trang web đọc truyện được dựng lên.

Là các độc giả truyện tranh lâu năm, các thành viên trong nhóm đều đã trải nghiệm rất nhiều trang web đọc truyện hiện nay và cùng đưa đến một quan điểm chung: các trang web đọc truyện tại Việt Nam vẫn còn tương đối giống nhau và có các điểm yếu cần được cải thiện:

\begin{itemize}
	\item Các trang web đang quan tâm tới số lượng hơn là chất lượng. Chất lượng hình ảnh chưa được thật sự tốt, các nhóm dịch tự do up truyện kém chất lượng.
	\item Giao diện xấu, thiếu tính nhất quán, nhiều quảng cáo, chưa đem lại được sự tập trung, nguồn cảm hứng cho người đọc.
	\item Thường chỉ cung cấp một ngôn ngữ duy nhất (Tiếng Việt), rất nhiều độc giả khi muốn đọc truyện dưới nhiều ngôn ngữ khác nhau sẽ phải tìm đọc thêm ở các trang web khác.
\end{itemize}

Nhận thấy lĩnh vực web đọc truyện vẫn còn tiềm năng có thể khai thác, các thành viên trong nhóm đã đồng nhất với nhau về ý tưởng tạo ra một trang web có thể khắc phục các hạn chế nêu trên. Vừa đứng trên vị trí của một người đọc truyện tranh, vừa đứng trên vị trí của người phát triển sản phẩm, các thành viên tin rằng sản phẩm lần này của nhóm có thể đưa ra các cải thiện đột phá để có thể khắc phục các điểm còn thiếu sót trên các trang đọc truyện hiện nay.

Vì vậy nhóm đã quyết định tạo ra dự án web đọc truyện tranh \textbf{Honyomi}.

\end{document}