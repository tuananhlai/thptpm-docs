% \documentclass[12pt, a4paper]{article}
\documentclass[12pt,a4paper]{report}
\usepackage[utf8]{vietnam}
\usepackage[T5]{fontenc}

\usepackage{amsmath}
\usepackage{amsfonts}
\usepackage{amssymb}
\usepackage{makeidx}
\usepackage{imakeidx}
\usepackage{graphicx}
\usepackage{graphics}
\usepackage{placeins}
\usepackage[unicode, bookmarksopenlevel=4]{hyperref}
\usepackage{makeidx}
\usepackage[style=alphabetic]{biblatex}
\usepackage{multicol}
\usepackage{subfiles}
\usepackage{hyperref}
\usepackage{enumitem}
\usepackage{float}
\usepackage[table,xcdraw]{xcolor}
\usepackage{tabularx}
\usepackage{wrapfig}
\usepackage{caption}
\usepackage{subcaption}
\usepackage{placeins}
\usepackage{array}
\usepackage{longtable}
\usepackage{multirow}
\usepackage{tikz}
\usepackage{pgfplots}
\usetikzlibrary{calc}

\let\orgautoref\autoref
\def\code#1{\texttt{#1}}

\setcounter{secnumdepth}{4}
\setcounter{tocdepth}{2}

\newcommand{\iindex}[1]{\textit{#1}\index{#1}}

% Create file reference.bib to add
\addbibresource{./reference.bib}

\graphicspath{ {./images/} {./../images}}
\DeclareGraphicsExtensions{.png,.eps,.svg}
\setlist[description]{leftmargin=\parindent,labelindent=\parindent}

\title{Hệ thống quản lý thực tập UETWork}

\pagenumbering{roman}
\begin{document}

\subfile{cover.tex}
\clearpage{}

\chapter*{Tóm tắt}

Hiện nay, kết nối sinh viên với nhà tuyển dụng là một nhu cầu thiết yếu của trường Đại học Công Nghệ. Đối với sinh viên, thực tập là một trải nghiệm quý báu, giúp họ áp dụng kiến thức đã học vào dự án thực tế và phát triển kỹ năng mềm như kỹ năng giao tiếp và làm việc nhóm. Mặt khác, nhà tuyển dụng luôn có nhu cầu tìm kiếm sinh viên tài năng để tham gia vào dự án phần mềm của mình.

Để giải quyết nhu cầu này, trường Đại học Công Nghệ đã phát triển hệ thống thực tập UETWork \footnote{\url{http://112.137.129.69:8000}}. Tuy nhiên, hệ thống này đã thể hiện một số nhược điểm trong thời gian được sử dụng như thiếu tính năng cho giảng viên và công ty, khó sử dụng, \ldots{} Hệ thống UETWork mới được mô tả trong khóa luận này đã được phát triển để giải quyết những vấn đề đó.

\chapter*{Lời cảm ơn}

Đầu tiên, em muốn gửi lời cảm ơn tới TS\. Dương Lê Minh, người đã giúp đỡ em trong quá trình hoàn thành báo cáo này.
Em cũng xin gửi lời cảm ơn chân thành đến Ban giám hiệu trường Đại học Công nghệ - ĐHQGHN, quý thầy cô khoa Công Nghệ Thông Tin đã tận tâm giảng dạy cho em những kiến thức cần thiết để em có thể hoàn thành tốt khóa luận tốt nghiệp này.
Vì kiến thức chuyên môn của bản thân còn nhiều hạn chế, khóa luận của em có thể vẫn còn một vài điểm chưa được hoàn hảo, kính mong nhà trường và thầy cô thông cảm và đóng góp ý kiến giúp em hoàn thiện hơn.

\chapter*{Lời cam đoan}

Em xin cam đoan rằng khóa luận tốt nghiệp này không sao chép từ bất kỳ ai, tổ chức nào khác. Toàn bộ nội dung được trình bày trong tài liệu đều là cá nhân em qua quá trình học tập, tìm hiểu và nghiên cứu mà hoàn thiện. Mọi tài liệu tham khảo đều được ghi chép lại và trích dẫn hợp pháp. Nếu lời cam đoan là sai sự thật thì em xin chịu mọi trách nhiệm và hình thức kỷ luật theo quy định từ phía nhà trường.

\tableofcontents{}
\clearpage{}

\listoffigures{}

\listoftables{}

\chapter{Mở đầu}
\pagenumbering{arabic}

\section{Đặt vấn đề}
\subfile{./sections/section_1.tex}

\chapter{Phân tích và đặc tả yêu cầu}

\section{Phân tích yêu cầu}
\subfile{./sections/section_2.tex}

\section{Đặc tả ca sử dụng}
\subfile{./sections/section_5.tex}

\chapter{Thiết kế hệ thống}

\section{Kiến thức cơ sở}
\subfile{./sections/section_3.tex}

\section{Thiết kế hệ thống}
\subfile{./sections/section_4.tex}

% \section{Thiết kế giao diện}
% \subfile{./sections/section_5.tex}

\chapter{Kết quả cài đặt và đánh giá}

\section{Tính năng nổi bật}
\subfile{./sections/section_6.tex}

\section{Kết quả kiểm thử}
\subfile{./sections/section_7.tex}

\section{Triển khai và kết quả thực nghiệm}
\subfile{./sections/section_8.tex}

\chapter{Kết luận và định hướng phát triển}
\subfile{./sections/section_9.tex}

\nocite{*}
\printbibliography[heading=bibintoc, title=Tài liệu tham khảo]

% \chapter*{Từ điển chú giải}

\end{document}

