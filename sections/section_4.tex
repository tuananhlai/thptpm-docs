% !TeX root = ../main.tex
\documentclass[./../main.tex]{subfiles}

\begin{document}

\subsection{Kiến trúc}

Hình \ref{fig:high_level_design} mô tả thiết kế bậc cao của hệ thống UETWork mới. Hệ thống bao gồm 4 thành phần chính:

\begin{itemize}
\item
  
  \textbf{Web App}: là ứng dụng web mà người dùng tương tác để hoàn
  thành công việc của mình.
  
\item
  
  \textbf{API Gateway}: nhận yêu cầu HTTP từ client, gửi yêu cầu tới
  những service cần thiết, sau đó tổng hợp lại response để trả lại kết
  quả cho client.
  
\item
  
  \textbf{User Service}: xử lý yêu cầu liên quan đến account và profile
  của người dùng trên hệ thống, bao gồm sinh viên, giảng viên, đối tác,
  \ldots{} cùng với thông tin về các khoa. Service này nhận yêu cầu bằng
  giao thức gRPC và HTTP (cho API upload file).
  
\item
  
  \textbf{Internship Service}: xử lý yêu cầu liên quan đến kỳ thực tập,
  ví dụ như danh sách kỳ thực tập, các sinh viên trong kỳ, \ldots{}
  Service này sẽ replicate lại một số data từ User Service để đảm bảo
  tính đúng đắn và toàn vẹn của dữ liệu. Service này nhận yêu cầu bằng
  giao thức gRPC và HTTP (cho API upload file).
  
\end{itemize}

Ngoài 4 thành phần chính ở trên, hệ thống có các thành phần phụ là:

\begin{itemize}
\item
  
  \textbf{Debezium}: đọc bản ghi từ cơ sở dữ liệu của User Service,
  chuyển bản ghi đó thành Kafka message và gửi vào Kafka.
  
\item
  
  \textbf{Internship Service Replicator}: đọc bản ghi của User DB từ
  Kafka và insert vào Internship DB.
  
\end{itemize}

\begin{figure}
	\includegraphics[width=\linewidth]{./images/image6.png}
	\caption{Thiết kế bậc cao}
	\label{fig:high_level_design}
\end{figure}

\subsection{Thiết kế cơ sở dữ liệu}

Hệ thống UETWork có 2 service chính: User Service và Internship Service. Mỗi service có nhiệm vụ tự quản lý dữ liệu của mình. Dưới đây là lược đồ cơ sở dữ liệu cho 2 service này.

\subsubsection{Cơ sở dữ liệu của User Service (User DB)}

Hình \ref{fig:user_db_design} mô tả các bảng và mối quan hệ giữa chúng trong User DB. Bảng \ref{tab:db_roles} đến \ref{tab:db_partner_orgs} mô tả cụ thể hơn về nội dung từng bảng trong User DB.

\begin{figure}
	\includegraphics[width=\linewidth]{./images/image3.png}
	\caption{Lược đồ User DB}
	\label{fig:user_db_design}
\end{figure}

\begin{table}[H]
	\caption[Bảng roles]{Bảng \textbf{roles} - chứa thông tin về các role (vai trò)}
	\label{tab:db_roles}
	\begin{tabular}{|l|l|l|}
	\hline
	\textbf{Tên cột} & \textbf{Kiểu dữ liệu} & \textbf{Mô tả} \\ \hline
	id               & int                   & Mã định danh   \\ \hline
	name             & varchar               & Tên của role   \\ \hline
	phone\_number    & varchar               & Số điện thoại  \\ \hline
	\end{tabular}
\end{table}

\begin{table}[H]
	\caption[Bảng users]{Bảng \textbf{users} - chứa thông tin về tài khoản người dùng}
	\label{tab:db_users}
	\begin{tabularx}{\textwidth}{|l|l|X|}
	\hline
	\textbf{Tên cột}    & \textbf{Kiểu dữ liệu} & \textbf{Mô tả}                                         \\ \hline
	id                  & int                   & Mã định danh                                           \\ \hline
	avatar\_url         & varchar               & Địa chỉ avatar của người dùng                          \\ \hline
	org\_email          & varchar               & Địa chỉ email để người dùng đăng nhập                  \\ \hline
	role\_id            & int                   & Mã định danh của role                                  \\ \hline
	password            & varchar               & Mật khẩu đã được băm bằng thuật toán bcrypt            \\ \hline
	last\_login\_at     & timestamp             & Thời gian đăng nhập thành công gần nhất của người dùng \\ \hline
	is\_email\_verified & boolean               & Email người dùng đã được xác nhận hay chưa?            \\ \hline
	created\_at         & timestamp             & Thời gian tạo                                          \\ \hline
	updated\_at         & timestamp             & Thời gian cập nhật gần nhất                            \\ \hline
	\end{tabularx}
\end{table}

\begin{table}[H]
	\caption[Bảng organizations]{Bảng \textbf{organizations} - chứa thông tin về các khoa}
	\label{tab:db_organizations}
	\begin{tabular}{|l|l|l|}
	\hline
	\textbf{Tên cột} & \textbf{Kiểu dữ liệu} & \textbf{Mô tả}                      \\ \hline
	id               & int                   & Mã định danh                        \\ \hline
	name             & varchar               & Tên khoa                            \\ \hline
	admin\_id        & int                   & Mã định danh của quản trị viên khoa \\ \hline
	created\_at      & timestamp             & Thời gian tạo                       \\ \hline
	\end{tabular}
\end{table}

\begin{table}[H]
	\caption[Bảng school\_classes]{Bảng \textbf{school\_classes} - chứa thông tin về lớp khóa học}
	\label{tab:db_school_classes}
	\begin{tabular}{|l|l|l|}
	\hline
	\textbf{Tên cột} & \textbf{Kiểu dữ liệu} & \textbf{Mô tả}           \\ \hline
	id               & int                   & Mã định danh             \\ \hline
	name             & varchar               & Tên lớp khóa học         \\ \hline
	program\_name    & varchar               & Tên chương trình đào tạo \\ \hline
	\end{tabular}
\end{table}

\begin{table}[H]
	\caption[Bảng students]{Bảng \textbf{students} - chứa thông tin về profile sinh viên}
	\label{tab:db_students}
	\begin{tabularx}{\textwidth}{|l|l|X|}
	\hline
	\textbf{Tên cột}    & \textbf{Kiểu dữ liệu} & \textbf{Mô tả}                        \\ \hline
	id                  & int                   & Mã định danh                          \\ \hline
	student\_id\_number & varchar               & Mã số sinh viên (được nhà trường cấp) \\ \hline
	full\_name          & varchar               & Họ tên đầy đủ                         \\ \hline
	resume\_url         & varchar               & Địa chỉ lưu trữ CV của người dùng     \\ \hline
	phone\_number       & varchar               & Số điện thoại của người dùng          \\ \hline
	school\_class\_id   & int                   & Mã định danh lớp học của sinh viên    \\ \hline
	personal\_email     & varchar               & Địa chỉ email cá nhân của sinh viên   \\ \hline
	created\_at         & timestamp             & Thời gian tạo                         \\ \hline
	updated\_at         & timestamp             & Thời gian cập nhật gần nhất           \\ \hline
	organization\_id    & int                   & Mã định danh khoa của sinh viên       \\ \hline
	user\_id            & int                   & Mã định danh tài khoản                \\ \hline
	org\_email          & varchar               & Email VNU                             \\ \hline
	\end{tabularx}%
\end{table}

\begin{table}[H]
	\caption[Bảng lecturers]{Bảng \textbf{lecturers} - chứa thông tin về profile giảng viên}
	\label{tab:db_lecturers}
	\begin{tabular}{|l|l|l|}
	\hline
	\textbf{Tên cột} & \textbf{Kiểu dữ liệu} & \textbf{Mô tả}          \\ \hline
	id               & int                   & Mã định danh            \\ \hline
	full\_name       & varchar               & Họ tên đầy đủ           \\ \hline
	org\_email       & varchar               & Địa chỉ email VNU       \\ \hline
	personal\_email  & varchar               & Địa chỉ email cá nhân   \\ \hline
	phone\_number    & varchar               & Số điện thoại           \\ \hline
	organization\_id & int                   & Mã định danh khoa       \\ \hline
	user\_id         & int                   & Mã định danh người dùng \\ \hline
	\end{tabular}%
\end{table}

\begin{table}[H]
	\caption[Bảng partners]{Bảng \textbf{partners} - chứa thông tin về profile công ty trong hệ thống}
	\label{tab:db_partners}
	\begin{tabularx}{\textwidth}{|l|l|X|}
	\hline
	\textbf{Tên cột} & \textbf{Kiểu dữ liệu} & \textbf{Mô tả}                               \\ \hline
	id               & int                   & Mã định danh                                 \\ \hline
	name             & varchar               & Tên công ty                                  \\ \hline
	homepage\_url    & varchar               & Địa chỉ trang chủ công ty                    \\ \hline
	logo\_url        & varchar               & Địa chỉ logo công ty                         \\ \hline
	description      & varchar               & Mô tả công ty                                \\ \hline
	user\_id         & int                   & Mã định danh người dùng                      \\ \hline
	email            & varchar               & Địa chỉ email của bộ phận tuyển dụng công ty \\ \hline
	created\_at      & timestamp             & Thời gian tạo                                \\ \hline
	updated\_at      & timestamp             & Thời gian cập nhật gần nhất                  \\ \hline
	phone\_number    & varchar               & Số điện thoại                                \\ \hline
	address          & varchar               & Địa chỉ công ty                              \\ \hline
	org\_email       & varchar               & Email VNU                                    \\ \hline
	\end{tabularx}
\end{table}

\begin{table}[H]
	\caption[Bảng partner\_contacts]{Bảng \textbf{partner\_contacts} - chứa thông tin về liên hệ của đối tác}
	\label{tab:db_partner_contacts}
	\begin{tabular}{|l|l|l|}
	\hline
	\textbf{Tên cột} & \textbf{Kiểu dữ liệu} & \textbf{Mô tả}              \\ \hline
	id               & int                   & Mã định danh                \\ \hline
	full\_name       & varchar               & Họ tên đầy đủ               \\ \hline
	phone\_number    & varchar               & Số điện thoại               \\ \hline
	partner\_id      & int                   & Mã định danh công ty        \\ \hline
	email            & varchar               & Địa chỉ email của liên hệ   \\ \hline
	created\_at      & timestamp             & Thời gian tạo               \\ \hline
	updated\_at      & timestamp             & Thời gian cập nhật gần nhất \\ \hline
	deleted\_at      & timestamp             & Thời gian liên hệ bị xóa    \\ \hline
	\end{tabular}
\end{table}

\begin{table}[H]
	\caption[Bảng partner\_organizations]{Bảng \textbf{partner\_organizations} - chứa trạng thái liên kết giữa công ty và khoa}
	\label{tab:db_partner_orgs}
	\begin{tabular}{|l|l|l|}
	\hline
	\textbf{Tên cột} & \textbf{Kiểu dữ liệu} & \textbf{Mô tả}                       \\ \hline
	id               & int                   & Mã định danh                         \\ \hline
	partner\_id      & int                   & Mã định danh công ty                 \\ \hline
	organization\_id & int                   & Mã định danh khoa                    \\ \hline
	expired\_date    & date                  & Thời gian kết thúc hợp đồng liên kết \\ \hline
	created\_at      & timestamp             & Thời gian tạo                        \\ \hline
	\end{tabular}
\end{table}

Bảng \textbf{migrations} chứa thông tin liên quan đến migration của cơ sở dữ liệu. Không liên quan tới chức năng của hệ thống

\subsubsection{Cơ sở dữ liệu của Internship Service}

Hình \ref{fig:internship_db_design} mô tả thiết kế của Internship DB. Cơ sở dữ liệu này đã nhân bản lại một số bảng từ User DB để giúp việc query dễ dàng hơn. Các bảng được nhân bản từ User DB bao gồm:

\begin{itemize}
\item
  
  Bảng students
  
\item
  
  Bảng lecturers
  
\item
  
  Bảng partners
  
\item
  
  Bảng partner\_contacts
  
\item
  
  Bảng organizations
  
\item
  
  Bảng partner\_organizations
  
\item
  
  Bảng school\_classes
  
\end{itemize}

Bảng \ref{tab:db_terms} đến \ref{tab:db_posts} mô tả nội dung cụ thể của từng bảng trong cơ sở dữ liệu Internship DB.

\begin{figure}
	\includegraphics[width=\linewidth]{./images/image1.png}
	\caption{Lược đồ Internship DB}
	\label{fig:internship_db_design}
\end{figure}

\begin{table}[H]
	\caption[Bảng internship\_terms]{Bảng \textbf{internship\_terms} - chứa kỳ thực tập của các khoa}
	\label{tab:db_terms}
	\begin{tabular}{|l|l|l|}
	\hline
	\textbf{Tên cột} & \textbf{Kiểu dữ liệu} & \textbf{Mô tả}                     \\ \hline
	id               & int                   & Mã định danh                       \\ \hline
	year             & year                  & Năm của kỳ thực tập                \\ \hline
	term             & int                   & Đợt của kỳ thực tập (v.d. 1, 2, 3) \\ \hline
	start\_reg\_at   & timestamp             & Thời gian bắt đầu đăng ký          \\ \hline
	end\_reg\_at     & timestamp             & Thời gian kết thúc đăng ký         \\ \hline
	start\_date      & date                  & Ngày bắt đầu kỳ thực tập           \\ \hline
	end\_date        & date                  & Ngày kết thúc kỳ thực tập          \\ \hline
	created\_at      & timestamp             & Thời gian tạo                      \\ \hline
	organization\_id & int                   & Mã định danh khoa                  \\ \hline
	\end{tabular}
\end{table}

\begin{table}[H]
	\caption[Bảng term\_students]{Bảng \textbf{term\_students} - chứa thông tin thực tập của toàn bộ sinh viên}
	\label{tab:db_term_students}
	\begin{tabular}{|l|l|l|}
	\hline
	\textbf{Tên cột}      & \textbf{Kiểu dữ liệu} & \textbf{Mô tả}                       \\ \hline
	term\_id              & int                   & Mã định danh kỳ thực tập             \\ \hline
	student\_id           & int                   & Mã định danh sinh viên               \\ \hline
	supervisor\_id        & int                   & Mã định danh giảng viên hướng dẫn    \\ \hline
	score                 & int                   & Điểm                                 \\ \hline
	created\_at           & timestamp             & Thời gian tạo                        \\ \hline
	updated\_at           & timestamp             & Thời gian cập nhật gần nhất          \\ \hline
	selected\_partner\_id & int                   & Mã định danh công ty được chọn       \\ \hline
	selected\_at          & timestamp             & Thời gian chọn công ty               \\ \hline
	report\_file\_name    & varchar               & Địa chỉ lưu trữ file báo cáo cuối kỳ \\ \hline
	\end{tabular}
\end{table}



\begin{table}[H]
	\caption[Bảng term\_lecturers]{Bảng \textbf{term\_lecturers} - chứa thông tin thực tập của giảng viên}
	\label{tab:db_term_lecturers}
	\begin{tabular}{|l|l|l|}
	\hline
	\textbf{Tên cột} & \textbf{Kiểu dữ liệu} & \textbf{Mô tả}           \\ \hline
	term\_id         & int                   & Mã định danh kỳ thực tập \\ \hline
	lecturer\_id     & int                   & Mã định danh giảng viên  \\ \hline
	created\_at      & timestamp             & Thời gian tạo            \\ \hline
	\end{tabular}
\end{table}


\begin{table}[H]
	\caption[Bảng term\_partners]{Bảng \textbf{term\_partners} - chứa thông tin của công ty trong kỳ thực tập}
	\label{tab:db_term_partners}
	\begin{tabularx}{\textwidth}{|l|X|X|}
	\hline
	\textbf{Tên cột} & \textbf{Kiểu dữ liệu}                   & \textbf{Mô tả}                                               \\ \hline
	term\_id         & int                                     & Mã định danh kỳ thực tập                                     \\ \hline
	partner\_id      & int                                     & Mã định danh công ty                                         \\ \hline
	status           & enum(‘ACCEPTED', ‘PENDING', ‘REJECTED’) & Trạng thái công ty (được chấp nhận / chờ duyệt / bị từ chối) \\ \hline
	created\_at      & timestamp                               & Thời gian tạo                                                \\ \hline
	updated\_at      & timestamp                               & Thời gian cập nhật gần nhất                                  \\ \hline
	\end{tabularx}
\end{table}


\begin{table}[H]
	\caption[Bảng term\_partner\_registrations]{Bảng \textbf{term\_partner\_registrations} - chứa thông tin đăng ký thực tập của sinh viên}
	\label{tab:db_term_partner_regs}
	\begin{tabularx}{\textwidth}{|l|X|X|}
	\hline
	\textbf{Tên cột}     & \textbf{Kiểu dữ liệu}                           & \textbf{Mô tả}                                 \\ \hline
	id                   & int                                             & Mã định danh                                   \\ \hline
	term\_id             & int                                             & Mã định danh kỳ thực tập                       \\ \hline
	student\_id          & int                                             & Mã định danh sinh viên                         \\ \hline
	partner\_id          & int                                             & Mã định danh công ty                           \\ \hline
	status               & enum(‘PASSED', ‘FAILED', ‘SELECTED', ‘WAITING') & Kết quả phỏng vấn                              \\ \hline
	is\_self\_registered & boolean                                         & Có phải sinh viên này đã tự đăng ký không?     \\ \hline
	created\_at          & timestamp                                       & Thời gian tạo                                  \\ \hline
	updated\_at          & timestamp                                       & Thời gian cập nhật gần nhất                    \\ \hline
	is\_mail\_sent       & boolean                                         & Mail thông báo đã được gửi hay chưa?           \\ \hline
	internship\_type     & enum(‘ASSOCIATE’, ‘OTHER')                      & Loại thực tập (thực tập đối tác/công ty ngoài) \\ \hline
	\end{tabularx}
\end{table}

\begin{table}[H]
	\caption[Bảng recruitment\_posts]{Bảng \textbf{recruitment\_posts} - chứa bài đăng tuyển dụng trong 1 kỳ thực tập}
	\label{tab:db_posts}
	\begin{tabular}{|l|l|l|}
	\hline
	\textbf{Tên cột}     & \textbf{Kiểu dữ liệu} & \textbf{Mô tả}                   \\ \hline
	id                   & int                   & Mã định danh                     \\ \hline
	term\_id             & int                   & Mã định danh kỳ thực tập         \\ \hline
	partner\_id          & int                   & Mã định danh công ty             \\ \hline
	partner\_contact\_id & int                   & Mã định danh liên hệ của đối tác \\ \hline
	start\_reg\_at       & timestamp             & Ngày bắt đầu đăng ký             \\ \hline
	end\_reg\_at         & timestamp             & Ngày kết thúc đăng ký            \\ \hline
	title                & varchar               & Tiêu đề bài đăng                 \\ \hline
	content              & varchar               & Nội dung bài đăng                \\ \hline
	job\_count           & int                   & Số người cần tuyển               \\ \hline
	\end{tabular}
\end{table}

\subsection{Các bài toán khó và hướng giải quyết}

Phần này mô tả những bài toán khó em gặp phải khi cài đặt hệ thống với
kiến trúc microservice và hướng giải quyết cho những bài toán đó.

\hypertarget{cuxe1ch-ux111ux1ecdc-dux1eef-liux1ec7u-nux1eb1m-ux1edf-2-cux1a1-sux1edf-dux1eef-liux1ec7u-khuxe1c-nhau}{%
\subsubsection{Cách đọc dữ liệu nằm ở 2 cơ sở dữ liệu khác
nhau}\label{cuxe1ch-ux111ux1ecdc-dux1eef-liux1ec7u-nux1eb1m-ux1edf-2-cux1a1-sux1edf-dux1eef-liux1ec7u-khuxe1c-nhau}}

Một hệ thống luôn cần có khả năng đọc dữ liệu từ nhiều bảng khác nhau để
phục vụ cho những mục đích như thống kê, \ldots{} Tuy nhiên, với kiến
trúc Microservice, mỗi service sẽ sở hữu và quản lý data của mình. Vì
vậy, việc đọc từ nhiều bảng dữ liệu khác nhau trở nên khó khăn hơn, vì
các bảng có thể nằm trên 2 cơ sở dữ liệu khác nhau.

Để giải quyết vấn đề này, em đã cân nhắc những giải pháp sau:

\begin{enumerate}
\def\labelenumi{\arabic{enumi}.}
\item
  
  Gọi API tới cả 2 service và ghép kết quả trả về thủ công
  
\item
  
  Nhân bản data giữa 2 service để giúp service thực hiện query trên 1 cơ
  sở dữ liệu duy nhất
  
\end{enumerate}

Ưu điểm của cách 1 là sự phân chia rõ ràng trong nhiệm vụ của từng
service. Tuy nhiên, nhược điểm của cách làm này là sự phức tạp trong quá
trình đọc và ghép dữ liệu từ 2 service, cũng như tạo ra một sự phụ thuộc
giữa 2 service vào nhau.

Cách 2 có ưu điểm là service có thể query dữ liệu đơn giản hơn (vì dữ
liệu cùng nằm trong 1 cơ sở dữ liệu), đồng thời giúp 2 service chạy độc
lập hơn. Nhược điểm của cách làm này là hệ thống cần cài đặt cơ chế nhân
bản dữ liệu giữa 2 service, và cần đảm bảo dữ liệu nhân bản là chính xác
và kịp thời.

Với ưu điểm và nhược điểm kể trên, em đã chọn cách số 2 để giải quyết
vấn đề này.

\hypertarget{cuxe1ch-cuxe0i-ux111ux1eb7t-tuxednh-nux103ng-xuxe1c-thux1ef1c-vuxe0-phuxe2n-quyux1ec1n}{%
\subsubsection{Cách cài đặt tính năng xác thực và phân
quyền}\label{cuxe1ch-cuxe0i-ux111ux1eb7t-tuxednh-nux103ng-xuxe1c-thux1ef1c-vuxe0-phuxe2n-quyux1ec1n}}

Với hệ thống theo kiến trúc monolith, ta có thể phân quyền các route một
cách trực tiếp. Tuy nhiên, với kiến trúc Microservice, việc cài đặt phân
quyền trong từng service sẽ tăng độ phức tạp lên đáng kể.

Để giải quyết vấn đề này, em quyết định giao nhiệm vụ xác thực cho một
service duy nhất. Service này có nhiệm vụ xác thực người dùng và trả về
token. Sau đó, em bổ sung tính năng phân quyền vào API Gateway để đảm
bảo người dùng không truy cập trái phép.

Do service không cài đặt tính năng phân quyền, em đã bổ sung tính năng
xác thực bằng API Key. Chỉ những yêu cầu chứa API Key đúng mới có thể
được xử lý bởi service. Như vậy, dù service có bị expose ra ngoài thì
các bên thứ ba cũng không thể gửi request đến service được.

\hypertarget{cho-phuxe9p-nhiux1ec1u-khoa-sux1eed-dux1ee5ng-hux1ec7-thux1ed1ng}{%
\subsubsection{Cho phép nhiều khoa sử dụng hệ
thống}\label{cho-phuxe9p-nhiux1ec1u-khoa-sux1eed-dux1ee5ng-hux1ec7-thux1ed1ng}}

Hệ thống UETWork mới có nhiệm vụ hỗ trợ sinh viên và admin của nhiều
khoa. Vì vậy, em cần thiết kế một giải pháp để phân tách dữ liệu của các
khoa, đảm bảo rằng các khoa không thể đọc dữ liệu của nhau.

Em đề ra 2 giải pháp cho vấn đề này:

\begin{itemize}
\item
  
  Tạo ra một service mới là Organization Service. Service này sẽ quản lý
  tất cả các khoa trên hệ thống. Dữ liệu của mỗi khoa sẽ được lưu trong
  một cơ sở dữ liệu khác nhau.
  
\item
  
  Bổ sung bảng organization vào user service và thêm trường
  \code{organization\_id} cho dữ liệu thuộc 1 khoa. Khi người dùng cần đọc /
  thêm / sửa dữ liệu của 1 khoa, em sẽ kiểm tra quyền sở hữu trước khi
  cho phép thực hiện thao tác đó.
  
\end{itemize}

Cách 1 có ưu điểm là dữ liệu được phân tách rõ ràng và người dùng của
một khoa gần như không thể đọc dữ liệu của khoa khác. Nhược điểm của
cách làm này xuất hiện khi admin cần thống kê dữ liệu từ tất cả các
khoa. Hệ thống sẽ cần phải đọc và tổng hợp từ nhiều cơ sở dữ liệu khác
nhau, khiến quá trình này phức tạp hơn.

Cách 2 có ưu điểm là việc đọc và tổng hợp dữ liệu sẽ dễ dàng hơn. Nhược
điểm của cách làm này là tất cả dữ liệu thuộc 1 khoa đều phải có trường
organization\_id và yêu cầu đọc / thêm / sửa dữ liệu phải được kiểm tra
để tránh truy cập trái phép.

Với ưu và nhược điểm kể trên, em đã chọn cách số 2 để giải quyết vấn đề
này.

\hypertarget{thiux1ebft-kux1ebf-api-ux111ux1ec3-client-dux1ec5-sux1eed-dux1ee5ng}{%
\subsubsection{Thiết kế API để client dễ sử
dụng}\label{thiux1ebft-kux1ebf-api-ux111ux1ec3-client-dux1ec5-sux1eed-dux1ee5ng}}

Với kiến trúc monolith, phía client có thể gửi yêu cầu đến server dễ
dàng do server chỉ có 1 địa chỉ duy nhất. Tuy nhiên, với kiến trúc
microservice, ta gặp những vấn đề sau:

\begin{itemize}
\item
  
  Dữ liệu nằm trên nhiều service khác nhau, yêu cầu nhà phát triển phía
  client phải nhớ dữ liệu nào nằm trên service nào trước khi gửi yêu
  cầu.
  
\item
  
  Một số yêu cầu cần dữ liệu nằm trên nhiều service khác nhau, yêu cầu
  nhà phát triển phía client phải gửi yêu cầu tới nhiều service và ghép
  lại một cách thủ công.
  
\end{itemize}

Để giải quyết vấn đề này, em đã áp dụng mẫu thiết kế API Gateway. API
Gateway là một server nằm ở giữa client và service. Khi nhận được yêu
cầu từ client, API Gateway sẽ gửi các yêu cầu cần thiết đến service, tập
hợp lại kết quả và gửi về kết quả. Cách làm này giúp đơn giản hóa quá
trình phát triển phía client. Nhà phát triển giờ đây chỉ cần gọi tới 1
địa chỉ và hiển thị kết quả, thay vì phải xử lý kết quả từ nhiều service
một cách thủ công như trước.

Hơn nữa, mẫu thiết kế API Gateway cho phép các service giao tiếp với
nhau một cách tối ưu hơn. Các service trong hệ thống UETWork mới giao
tiếp với nhau bằng giao thức gRPC thay vì HTTP, giúp giảm tải băng thông
và độ trễ cho các yêu cầu.

\end{document}
