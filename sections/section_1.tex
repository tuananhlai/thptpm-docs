% !TeX root = ../main.tex
\documentclass[./../main.tex]{subfiles}

\begin{document}

\subsection{Hiện trạng}

Hệ thống UETWork hiện tại được sinh ra để giúp sinh viên hoàn thành
chuyên đề thực tập tại trường của mình. Hệ thống là một nền tảng kết nối
giữa sinh viên và nhà tuyển dụng, giúp sinh viên tìm được nơi thực tập
chất lượng dễ dàng hơn.

Tuy vậy, trong vòng 2 năm vừa qua, hệ thống đã biểu hiện một số nhược
điểm như sau:

\begin{itemize}
\item
  
  Sinh viên được phép tạo công ty tùy ý, dẫn đến việc lặp dữ liệu.
  
\item
  
  Giao diện chưa rõ ràng, khó sử dụng cho người dùng.
  
\item
  
  Chi hỗ trợ cho người dùng của khoa Công Nghệ Thông Tin.
  
\item
  
  Không cung cấp tính năng cho giảng viên / công ty đối tác.
  
\item
  
  Chứa một số vấn đề về bảo mật / lỗi thiết kế. Ví dụ, hệ thống cho phép
  người dùng đặt lại mật khẩu cho bất kỳ người dùng nào khác trên hệ
  thống.
  
\item
  
  Mọi liên lạc với người dùng được thực hiện trong ứng dụng thay vì qua
  email, khiến người dùng phải kiểm tra ứng dụng thường xuyên hơn và
  tăng khả năng một thông báo bị bỏ qua.
  
\item
  
  Giao diện chưa hỗ trợ đa ngôn ngữ cho những sinh viên quốc tế.
  
\item
  
  Không có thống kê dữ liệu.
  
\item
  
  Thiếu bộ lọc dữ liệu cho các bảng.
  
\end{itemize}

Giải quyết vấn đề này đòi hỏi việc tái cấu trúc toàn bộ hệ thống thực tập hiện tại. Vì vậy,  việc thiết kế và phát triển một hệ thống mới kế thừa các ưu điểm và cải tiến các nhược điểm của hệ thống cũ là một nhu cầu thiết yếu hiện tại tại Khoa CNTT.

\subsection{Nhu cầu quản lý thực tập tại trường Đại học Công Nghệ}

Nhu cầu của nhà trường đối với hệ thống thực tập mới là như sau:

\begin{itemize}
\item
  
  Hỗ trợ người dùng của nhiều khoa khác nhau.
  
\item
  
  Cần tính năng chấm điểm / quản lý sinh viên cho người dùng giảng viên.
  
\item
  
  Cần tính năng quản lý sinh viên / chấp nhận yêu cầu thực tập cho người
  dùng đối tác.
  
\item
  
  Cần có giao diện và API để thống kê dữ liệu.
  
\item
  
  Cần bộ lọc dữ liệu đầy đủ trong màn hình quản lý.
\item Cần thông báo về các sự kiện trên hệ thống cho người dùng. Ví dụ, sinh viên cần được biết khi có kết quả phỏng vấn để họ có thể thực hiện các hành động tiếp theo.
\end{itemize}

Ngoài các nhu cầu kể trên, hệ thống mới cần đem lại những cải thiện sau:
\begin{itemize}
  \item Giao diện cần được thiết kế dễ hiểu hơn, giúp người dùng hoàn thành
  công việc trong thời gian nhanh nhất có thể.
  \item Dữ liệu phải được làm sạch. Ví dụ, tên công ty phải được duyệt bởi
    quản trị viên trước khi lưu vào cơ sở dữ liệu
  \item Cần hỗ trợ đa ngôn ngữ cho sinh viên quốc tế.
\end{itemize}

\end{document}