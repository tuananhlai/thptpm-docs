% !TeX root = ../main.tex
\documentclass[./../main.tex]{subfiles}

\begin{document}

\hypertarget{kiux1ebfn-truxfac-microservice}{%
\subsection{Kiến trúc
Microservice}\label{kiux1ebfn-truxfac-microservice}}

Kiến trúc Microservice \cite{Ric22a} là kiến trúc mà các thành phần của hệ thống được phân tách thành các service con chạy độc lập với nhau.

Ưu điểm của kiến trúc này là độ tin cậy và khả năng mở rộng của hệ thống
sẽ được tăng lên do lỗi của một service sẽ không ảnh hưởng đến những
service còn lại. Hơn nữa, do mã nguồn trong 1 service có liên quan chặt
chẽ đến nhau, code trong một service sẽ đơn giản và dễ bảo trì hơn.

Nhược điểm của kiến trúc này là độ phức tạp của cả hệ thống sẽ tăng lên.
Thay vì sử dụng các hàm local, service phải giao tiếp với nhau bằng một
giao thức khác (như TCP) hoặc bằng một hệ thống truyền tin (như Kafka,
RabbitMQ, \ldots). Hơn nữa, mức sử dụng tài nguyên của hệ thống cũng
tăng lên do các service cần những hệ thống bổ trợ.

\hypertarget{nodejs-vuxe0-nestjs}{%
\subsection{NodeJS và NestJS}\label{nodejs-vuxe0-nestjs}}

NodeJS \cite{Dah22} là một nền tảng được xây dựng dựa trên V8 Javascript Engine của
Google. Nền tảng này cho phép nhà phát triển tận dụng được sự phổ biến
và linh hoạt của JavaScript để phát triển ứng dụng backend.

NodeJS có thể chạy trên nhiều nền tảng khác nhau như Windows, Linux và
Mac OS. Ngoài ra, kiến trúc hướng sự kiện và cơ chế non-blocking I/O của
NodeJS giúp ứng dụng thực thi nhanh và hiệu quả hơn.

NestJS \cite{Mys22} là một framework để phát triển ứng dụng backend bằng NodeJS.
NestJS đem lại nhiều ưu điểm so với NodeJS thuần như:

\begin{itemize}
\item Sử dụng TypeScript, một phiên bản nâng cao của JavaScript vớihỗ trợ cho kiểu dữ liệu tĩnh và các lớp hướng đối tượng.
\item
  Áp dụng nguyên lý Dependency Injection để nâng cao hiệu năng và giảm
  lượng tài nguyên sử dụng.
\item
  Cung cấp các công cụ để phát triển ứng dụng theo mô hình hướng đối
  tượng.
\end{itemize}

\hypertarget{grpc}{%
\subsection{gRPC}\label{grpc}}

gRPC \cite{Goo22} là một giao thức hiện đại, được phát triển bởi Google để cho phép
server backend giao tiếp với nhau. gRPC sử dụng HTTP/2 cùng với Protobuf
để encode dữ liệu nhỏ nhất có thể, giúp giảm thời gian truyền tải của
payload. Do vậy, gRPC rất phổ biến với mô hình microservice.

\hypertarget{kafka}{%
\subsection{Kafka}\label{kafka}}

Kafka \cite{Kaf22} là một hệ thống truyền tin theo hướng publish-subscribe phân tán.
Bên public dữ liệu được gọi là \emph{producer}, bên subscribe gọi là
\emph{consumer}.

Kafka có khả năng truyền một lượng lớn message theo thời gian thực và có
thể đảm bảo consumer sẽ nhận được message ít nhất 1 lần (at-least-one
guarantee).

Khác với hệ thống truyền tin khác, Kafka lưu tất cả message ở trong đĩa
thay vì trong bộ nhớ, giúp cho message không bị mất khi server xảy ra
lỗi. Hơn nữa, người dùng có thể cài đặt time-to-live chơ những message
này để tiết kiệm không gian lưu trữ.

Kafka thường được sử dụng như một phương thức giao tiếp giữa các service
trong kiến trúc microservice. Ngoài ra, người dùng có thể tận dụng các
plugin của Kafka để giải quyết các vấn đề như nhân bản dữ liệu, \ldots{}

\hypertarget{debezium}{%
\subsection{Debezium}\label{debezium}}

Debezium  \cite{Hat22} là một Kafka Connector có nhiệm vụ đọc và chuyển hóa log của cơ
sở dữ liệu thành các Kafka Message và gửi chúng vào Kafka. Debezium
thường được sử dụng để nhân bản dữ liệu từ các nguồn khác nhau.

\hypertarget{minio}{%
\subsection{MinIO}\label{minio}}

MinIO \cite{Min22} là một object storage server mã nguồn mở và implement public API
của Amazon S3. MinIO có thể được cài đặt để lưu trữ file trong các
khoảng thời gian khác nhau, tùy vào nhu cầu của người sử dụng.

\hypertarget{docker}{%
\subsection{Docker}\label{docker}}

Docker \cite{Docker22} là nền tảng phần mềm cho phép người dùng dựng, kiểm thử và triển
khai ứng dụng một cách nhanh chóng. Docker đóng gói phần mềm vào các đơn
vị tiêu chuẩn hóa được gọi là \emph{container}.

Một container có mọi thứ mà phần mềm cần để chạy, trong đó có thư viện,
công cụ hệ thống, mã nguồn và thời gian chạy. Bằng cách sử dụng Docker,
người dùng có thể nhanh chóng triển khai và thay đổi quy mô ứng dụng vào
bất kỳ môi trường nào và biết chắc rằng mã nguồn sẽ chạy được. Do những
đặc điểm này, Docker được sử dụng phổ biến trong các hệ thống với kiến
trúc Microservice.

\hypertarget{nginx}{%
\subsection{Nginx}\label{nginx}}

Nginx \cite{Nginx22} là một phần mềm web server mã nguồn mở nổi tiếng. Ban đầu nó dùng
để phục vụ web HTTP. Tuy nhiên, ngày nay nó cũng được dùng làm reverse
proxy, HTTP load balancer và email proxy như IMAP, POP3, và SMTP.

\hypertarget{certbot}{%
\subsection{Certbot}\label{certbot}}

Certbot là một công cụ của tổ chức Let's Encrypt để cung cấp và thu hồi
chứng chỉ SSL/TLS. Bằng việc sử dụng công cụ này kết hợp với web server
Nginx, người dùng có thể nâng cấp server của mình lên HTTPS một cách dễ
dàng và nhanh chóng.

\hypertarget{tuxedch-hux1ee3p-liuxean-tux1ee5c-cicd}{%
\subsection{Tích hợp liên tục
(CI/CD)}\label{tuxedch-hux1ee3p-liuxean-tux1ee5c-cicd}}

\emph{Continuous Integration/Continuous Delivery} hay CI/CD là một
phương pháp phát triển phần mềm. Phương pháp này đòi hỏi các thành viên
trong đội cần phải tích hợp công việc và triển khai code thường xuyên.
Mỗi lần tích hợp, hệ thống sẽ được xây dựng và triển khai một cách tự
động nhằm mục đích phát hiện ra những lỗi phát sinh một cách nhanh nhất
có thể.

CI/CD giúp làm giảm những vấn đề về tích hợp và triển khai, cho phép nhà
phát triển xây dựng phần mềm nhanh và hiệu quả hơn.

\hypertarget{gitlab-cicd}{%
\subsection{Gitlab CI/CD}\label{gitlab-cicd}}

Gitlab CI/CD \cite{Git22} là một dịch vụ dành cho việc áp dụng CI/CD vào trong quy
trình phát triển phần mềm. Dịch vụ này cho phép người dùng định nghĩa
các bước cần chạy sau khi git commit và cung cấp phần cứng để chạy những
bước này.

\end{document}