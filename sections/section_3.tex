% !TeX root = ../main.tex
\documentclass[./../main.tex]{subfiles}

\begin{document}

\hypertarget{typescript}{%
	\subsection{Typescript}\label{typescript}}

Typescript \cite{Type22} là một ngôn ngữ lập trình hướng đối tượng, một phiên bản cao hơn của Javascript, và được phát triển bởi Microsoft.

Bên cạnh các tính năng dùng để phát triển ứng dụng web  giống như Javascript, thì Typescript còn có một vài ưu điểm như:

\begin{itemize}
	\item

	      Static typing của Typescript hỗ trợ khai báo kiểu cho biến, và trình biên dịch sẽ giảm được tỷ lệ gán sai kiểu của các giá trị.

	\item
	      Interface trong Typescript sẽ đảm bảo đối tượng đi đôi với một cấu trúc nhất định, tránh sai sót dữ liệu.

	\item
	      Cách tổ chức code rõ ràng, hỗ trợ cơ chế module hóa, hỗ trợ namespace, giúp phát triển các hệ thống lớn nơi mà nhiều lập trình viên có thể làm việc cùng nhau một cách dễ dàng hơn.

\end{itemize}

\hypertarget{react}{%
	\subsection{React}\label{react}}

React \cite{React22} là thư viện JavaScript phổ biến nhất để xây dựng giao diện người
dùng (UI). Nó cho tốc độ phản hồi tuyệt vời khi user nhập liệu bằng cách
sử dụng phương pháp mới để render trang web.

Thư viện này được phát triển bởi Facebook. Mã nguồn của React được công
bố bởi Facebook vào năm 2013. Theo thống kê của StackOverFlow vào năm
2021, React là thư viện phổ biến nhất với 40.14\% người dùng, cao hơn
hẳn Angular (22.96\%) và Vue (18.97\%).

\hypertarget{react-material-ui}{%
	\subsection{React Material UI}\label{react-material-ui}}

React Material UI \cite{MUI22} là một thư viện component của React, được thiết kế
theo chuẩn Material Design của Google. Thư viện này cung cấp cho người
dùng một tập các thành phần giao diện hoàn chỉnh, giúp đẩy nhanh tiến độ
phát triển ứng dụng. Hơn nữa, Material UI cho phép người dùng tinh chỉnh
lại chủ đề, màu sắc, phông chữ của các thành phần để tạo ra một thiết kế
độc đáo của riêng mình.

\hypertarget{tuxedch-hux1ee3p-liuxean-tux1ee5c-cicd}{%
	\subsection{Tích hợp liên tục
		(CI/CD)}\label{tuxedch-hux1ee3p-liuxean-tux1ee5c-cicd}}

\emph{Continuous Integration/Continuous Delivery} hay CI/CD \cite{CICD22} là một
phương pháp phát triển phần mềm. Phương pháp này đòi hỏi các thành viên
trong đội cần phải tích hợp công việc và triển khai code thường xuyên.
Mỗi lần tích hợp, hệ thống sẽ được xây dựng và triển khai một cách tự
động nhằm mục đích phát hiện ra những lỗi phát sinh một cách nhanh nhất
có thể.

CI/CD giúp làm giảm những vấn đề về tích hợp và triển khai, cho phép nhà
phát triển xây dựng phần mềm nhanh và hiệu quả hơn.

\hypertarget{gitlab-cicd}{%
	\subsection{Gitlab CI/CD}\label{gitlab-cicd}}

Gitlab CI/CD \cite{Git22} là một dịch vụ dành cho việc áp dụng CI/CD vào trong quy
trình phát triển phần mềm. Dịch vụ này cho phép người dùng định nghĩa
các bước cần chạy sau khi git commit và cung cấp phần cứng để chạy những
bước này.

\end{document}