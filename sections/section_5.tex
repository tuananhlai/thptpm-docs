% !TeX root = ../main.tex
\documentclass[./../main.tex]{subfiles}

\begin{document}

\subsection{Đăng nhập hệ thống}

\paragraph*{Mô tả tóm tắt}

Người dùng đăng nhập vào hệ thống với tài khoản được cấp.

\paragraph*{Luồng chính} Ca sử dụng này bắt đầu khi người dùng mong muốn đăng
nhập vào hệ thống.

\begin{table}[H]
    \caption{Đăng nhập hệ thống - luồng chính}
    \label{tab:login_flow_main}
	\begin{tabularx}{\textwidth}{|X|X|X|}
		\hline
		\textbf{Hành động}                         & \textbf{Hệ thống phản hồi}                                                                      & \textbf{Dữ liệu}                   \\ \hline
		Người dùng truy cập trang đăng nhập. & Hệ thống yêu cầu người dùng nhập email và mật khẩu.                                   &                                        \\ \hline
		Người dùng điền các thông tin.        & Hệ thống kiểm định thông tin và điều hướng người dùng vào hệ thống theo Role. & Email và mật khẩu người dùng. \\ \hline
	\end{tabularx}
\end{table}

\paragraph*{Luồng thay thế}

\begin{itemize}
	\item
	      	Tại bước 3: Thay vì đăng nhập bằng email và mật khẩu, người dùng yêu
	      	cầu đăng nhập bằng Google, hệ thống điều hướng và mở một trang đăng
	      	nhập bằng tài khoản Google.
\end{itemize}

\paragraph*{Luồng ngoại lệ}

\begin{itemize}
	\item
	      	Tại bước 3: Người dùng điền thiếu hoặc sai thông tin, hệ thống yêu cầu
	      	nhập lại.
	\item
	      	Tại bước 4: Hệ thống kiểm định thông tin đăng nhập sai, hệ thống hiển
	      	thị thông tin lỗi, yêu cầu nhập lại. Người dùng có thể tiếp tục lặp
	      	lại luồng sự kiện hoặc thoát khỏi trang đăng nhập.
\end{itemize}

\paragraph*{Yêu cầu đặc biệt}

Người dùng đã có tài khoản trong hệ thống.

\paragraph*{Điều kiện đầu}

Không có.

\paragraph*{Điều kiện cuối}

Nếu ca sử dụng thành công, người dùng đã được đăng nhập vào hệ thống,
ngược lại, trạng thái hệ thống không thay đổi.

\paragraph*{Các vấn đề mở}

Không có.

\subsection{Đổi mật khẩu}

\paragraph*{Mô tả tóm tắt}

Người dùng thực hiện thay đổi mật khẩu tài khoản cá nhân.

\paragraph*{Luồng sự kiện}

\emph{Luồng chính:} Ca sử dụng bắt đầu khi người dùng đã đăng nhập thành
công và muốn thay đổi mật khẩu.

\emph{Luồng thay thế:} Không có

\emph{Luồng ngoại lệ:}

\begin{itemize}
\item
  \begin{quote}
  Tại bước 3: Người dùng nhập sai / thiếu thông tin hoặc mật khẩu mới
  không đáp ứng yêu cầu, hệ thống yêu cầu nhập lại.
  \end{quote}
\item
  \begin{quote}
  Tại bước 4: Hệ thống kiểm định thông tin sai / thiếu hoặc mật khẩu mới
  không đáp ứng yêu cầu, hệ thống yêu cầu nhập lại. Người dùng có thể
  tiếp tục lặp lại luồng sự kiện hoặc thoát khỏi trang hiện tại.
  \end{quote}
\end{itemize}

\paragraph*{Yêu cầu đặc biệt}

Không có.

\paragraph*{Điều kiện đầu}

Người dùng đã đăng nhập và hệ thống và yêu cầu thay đổi mật khẩu.

\paragraph*{Điều kiện cuối}

Nếu ca sử dụng thành công, người dùng đã thay đổi được mật khẩu, ngược
lại, trạng thái hệ thống không thay đổi.

\paragraph*{Các vấn đề mở}

Không có.

\subsection{Đặt lại mật khẩu}
\paragraph*{Mô tả tóm tắt}

Người dùng thực hiện đặt lại mật khẩu.

\paragraph*{Luồng sự kiện}

\emph{Luồng chính:} Ca sử dụng bắt đầu khi người dùng muốn đăng nhập vào
hệ thống nhưng quên mật khẩu.

\emph{Luồng thay thế:} Không có

\emph{Luồng ngoại lệ:}

\begin{itemize}
\item
  \begin{quote}
  Tại bước 3: Người dùng nhập vào email không tồn tại trong hệ thống. Hệ
  thống yêu cầu nhập lại.
  \end{quote}
\item
  \begin{quote}
  Tại bước 4: Hệ thống kiểm định email không tồn tại, hệ thống hiển thị
  thông báo cho người dùng, yêu cầu nhập lại. Người dùng có thể tiếp tục
  lặp lại luồng sự kiện hoặc thoát khỏi trang đăng nhập.
  \end{quote}
\end{itemize}

\paragraph*{Yêu cầu đặc biệt}

Người dùng đã có tài khoản trong hệ thống.

\paragraph*{Điều kiện đầu}

Không có.

\paragraph*{Điều kiện cuối}

Nếu ca sử dụng thành công, người dùng có thể thay đổi mật khẩu mới,
ngược lại, trạng thái hệ thống không thay đổi.

\paragraph*{Các vấn đề mở}

Không có.

\subsection{Đổi mật khẩu}

\paragraph*{Mô tả tóm tắt}

Người dùng thực hiện thay đổi mật khẩu tài khoản cá nhân.

\paragraph*{Luồng sự kiện}

\emph{Luồng chính:} Ca sử dụng bắt đầu khi người dùng đã đăng nhập thành công và muốn thay đổi mật khẩu.

\emph{Luồng thay thế:} Không có

\emph{Luồng ngoại lệ:}

\begin{itemize}
\item
  \begin{quote}
  Tại bước 3: Người dùng nhập sai / thiếu thông tin hoặc mật khẩu mới
  không đáp ứng yêu cầu, hệ thống yêu cầu nhập lại.
  \end{quote}
\item
  \begin{quote}
  Tại bước 4: Hệ thống kiểm định thông tin sai / thiếu hoặc mật khẩu mới
  không đáp ứng yêu cầu, hệ thống yêu cầu nhập lại. Người dùng có thể
  tiếp tục lặp lại luồng sự kiện hoặc thoát khỏi trang hiện tại.
  \end{quote}
\end{itemize}

\paragraph*{Yêu cầu đặc biệt}

Không có.

\paragraph*{Điều kiện đầu}

Người dùng đã đăng nhập và hệ thống và yêu cầu thay đổi mật khẩu.

\paragraph*{Điều kiện cuối}

Nếu ca sử dụng thành công, người dùng đã thay đổi được mật khẩu, ngược
lại, trạng thái hệ thống không thay đổi.

\paragraph*{Các vấn đề mở}

Không có.

\subsection{Chỉnh sửa thông tin cá nhân}

\paragraph*{Mô tả tóm tắt}

Người dùng yêu cầu thay đổi các thông tin cá nhân.

\paragraph*{Luồng sự kiện}

\emph{Luồng chính:} Ca sử dụng bắt đầu khi người dùng đã đăng nhập thành
công và muốn thay đổi thông tin cá nhân.

\emph{Luồng thay thế:} Không có.

\emph{Luồng ngoại lệ:} Không có.

\paragraph*{Yêu cầu đặc biệt}

Người dùng hệ thống là \textbf{sinh viên, giảng viên, đối tác.}

\paragraph*{Điều kiện đầu}

Người dùng đã đăng nhập hệ thống và mong muốn thay đổi thông tin cá
nhân.

\paragraph*{Điều kiện cuối}

Nếu ca sử dụng thành công, người dùng đã thay đổi được thông tin cá
nhân.

\paragraph*{Các vấn đề mở}

Không có

\paragraph*{Biểu đồ hoạt động}

\subsection{Xem thông tin bài đăng}

\paragraph*{Mô tả tóm tắt}

Người dùng yêu cầu xem thông tin chi tiết bài đăng tuyển dụng.

\paragraph*{Luồng sự kiện}

\emph{Luồng chính:} Ca sử dụng bắt đầu khi người dùng đã đăng nhập vào
hệ thống và muốn xem thông tin bài đăng.

\emph{Luồng thay thế:} Không có.

\emph{Luồng ngoại lệ:} Không có.

\paragraph*{Yêu cầu đặc biệt}

Người dùng là \textbf{sinh viên}.

\paragraph*{Điều kiện đầu}

Người dùng đã đăng nhập vào hệ thống và mong muốn xem thông tin bài đăng
tuyển dụng.

\paragraph*{Điều kiện cuối}

Nếu ca sử dụng thành công, người dùng xem được thông tin có trong bài
đăng.

\paragraph*{Các vấn đề mở}

Không có.

\subsection{Xem quy chế thực tập}

\paragraph*{Mô tả tóm tắt}

Người dùng lần đầu tiên đăng nhập vào hệ thống với tài khoản được cấp,
một email có nội dung về quy chế thực tập sẽ được gửi đến email người
dùng.

\paragraph*{Luồng sự kiện}

\emph{Luồng chính:}

\emph{Luồng thay thế:} Không có

\emph{Luồng ngoại lệ:}

\begin{itemize}
\item
  \begin{quote}
  Tại bước 1: Người dùng nhập sai / thiếu thông tin, hệ thống yêu cầu
  nhập lại.
  \end{quote}
\item
  \begin{quote}
  Tại bước 2: Hệ thống kiểm định thấy thông tin sai, hệ thống không gửi
  email, yêu cầu nhập lại. Người dùng có thể tiếp tục lặp lại luồng sự
  kiện hoặc thoát khỏi trang hiện tại.
  \end{quote}
\end{itemize}

\paragraph*{Yêu cầu đặc biệt}

Người dùng là \textbf{sinh viên} và lần đăng nhập này là đầu tiên.

\paragraph*{Điều kiện đầu}

Người dùng đăng nhập thành công vào hệ thống.

\paragraph*{Điều kiện cuối}

Nếu ca sử dụng thành công, người dùng sẽ nhận được email chứa nội dung
quy chế thực tập.

\paragraph*{Các vấn đề mở}

Không có.

\paragraph*{Biểu đồ hoạt động}

\subsection{Đăng ký thực tập với công ty liên kết}

\paragraph*{Mô tả tóm tắt}

Người dùng đã có nơi thực tập và là công ty đã liên kết với Khoa, mong
muốn đăng ký thực tập với công ty đó.

\paragraph*{Luồng sự kiện}

\emph{Luồng chính:}

\emph{Luồng thay thế:} Không có.

\emph{Luồng ngoại lệ:}

\begin{itemize}
\item
  \begin{quote}
  Tại bước 3: Người dùng chọn công ty đang ở trạng thái bị từ chối, hệ
  thống phản hồi thông báo.
  \end{quote}
\item
  \begin{quote}
  Tại bước 4: Hệ thống kiểm định thấy công ty đã chọn đang ở trạng thái
  bị từ chối, gửi lại thông báo cho người dùng. Người dùng có thể tiếp
  tục lặp lại luồng sự kiện hoặc thoát khỏi trang hiện tại.
  \end{quote}
\end{itemize}

\paragraph*{Yêu cầu đặc biệt}

Người dùng là \textbf{sinh viên}.

\paragraph*{Điều kiện đầu}

Người dùng đã đăng nhập thành công và mong muốn đăng ký thực tập tại
công ty đã liên kết.

\paragraph*{Điều kiện cuối}

Nếu ca sử dụng thành công, người dùng đã đăng ký thực tập thành công tại
công ty đã chọn, ngược lại, trạng thái hệ thống không thay đổi.

\paragraph*{Các vấn đề mở}

Không có.

\subsection{Đăng ký thực tập với công ty ngoài}

\paragraph*{Mô tả tóm tắt}

Người dùng đã có nơi thực tập và là công ty chưa liên kết với Khoa, mong
muốn đăng ký thực tập tại công ty đó.

\paragraph*{Luồng sự kiện}

\emph{Luồng chính:}

\emph{Luồng thay thế:} Không có.

\emph{Luồng ngoại lệ:}

\begin{itemize}
\item
  \begin{quote}
  Tại bước 3: Người dùng chọn công ty đang ở trạng thái bị từ chối, hệ
  thống phản hồi thông báo.
  \end{quote}
\item
  \begin{quote}
  Tại bước 4: Hệ thống kiểm định thấy công ty đã chọn đang ở trạng thái
  bị từ chối, gửi lại thông báo cho người dùng. Người dùng có thể tiếp
  tục lặp lại luồng sự kiện hoặc thoát khỏi trang hiện tại.
  \end{quote}
\end{itemize}

\paragraph*{Yêu cầu đặc biệt}

Người dùng là \textbf{sinh viên}.

\paragraph*{Điều kiện đầu}

Người dùng là đã đăng nhập thành công và mong muốn đăng ký thực tập tại
công ty ngoài đã có trong danh sách.

\paragraph*{Điều kiện cuối}

Nếu ca sử dụng thành công, người dùng đã đăng ký thực tập thành công tại
công ty đã chọn, ngược lại, trạng thái hệ thống không thay đổi.

\paragraph*{Các vấn đề mở}

Không có.

\subsection{Đăng ký phỏng vấn từ bài đăng}

\paragraph*{Mô tả tóm tắt}

Người dùng chưa có nơi thực tập, sau khi đọc bài đăng tuyển dụng trên hệ
thống, mong muốn đăng ký phỏng vấn thực tập tại các công ty đã đăng bài.

\paragraph*{Luồng sự kiện}

\emph{Luồng chính:}

\emph{Luồng thay thế:}

\begin{itemize}
\item
  \begin{quote}
  Tại bước 3: Người dùng không xem chi tiết bài đăng, mà yêu cầu đăng ký
  ngay trực tiếp ở trang chủ danh sách bài đăng.
  \end{quote}
\end{itemize}

\emph{Luồng ngoại lệ:} Không có.

\paragraph*{Yêu cầu đặc biệt}

Người dùng là \textbf{sinh viên}.

\paragraph*{Điều kiện đầu}

Người dùng đã đăng nhập thành công vào hệ thống và mong muốn đăng ký
phỏng vấn thực tập sau khi đọc bài đăng tuyển dụng.

\paragraph*{Điều kiện cuối}

Nếu ca sử dụng thành công, người dùng đã đăng ký phỏng vấn thành công
tại công ty đã chọn, ngược lại, trạng thái hệ thống không thay đổi.

\paragraph*{Các vấn đề mở}

Không có.

\subsection{Xem thông tin thực tập}

\paragraph*{Mô tả tóm tắt}

Người dùng mong muốn xem thông tin thực tập của bản thân.

\paragraph*{Luồng sự kiện}

\emph{Luồng chính:}

\emph{Luồng thay thế:} Không có.

\emph{Luồng ngoại lệ:} Không có.

\paragraph*{Yêu cầu đặc biệt}

Người dùng là \textbf{sinh viên}.

\paragraph*{Điều kiện đầu}

Người dùng đã đăng nhập vào hệ thống và mong muốn xem thông tin thực tập
của bản thân.

\paragraph*{Điều kiện cuối}

Nếu ca sử dụng thành công, người dùng đã có thể xem thông tin thực tập
của bản thân, ngược lại, trạng thái hệ thống không thay đổi.

\paragraph*{Các vấn đề mở}

Không có.

\subsection{Chọn công ty để thực tập}

\paragraph*{Mô tả tóm tắt}

Người dùng đã đăng ký thực tập thành công hoặc phỏng vấn đạt thì sau khi
kết thúc thời gian đăng ký, người dùng chọn một công ty duy nhất mà mình
đang và sẽ thực tập tại đó.

\paragraph*{Luồng sự kiện}

\emph{Luồng chính:}

\emph{Luồng thay thế:} Không có.

\emph{Luồng ngoại lệ:} Không có.

\paragraph*{Yêu cầu đặc biệt}

Người dùng là \textbf{sinh viên} và đã có ít nhất một công ty đang ở
trạng thái \emph{PASSED}.

\paragraph*{Điều kiện đầu}

Người dùng đã đăng nhập, đã truy cập vào trang thông tin thực tập và
mong muốn chọn một công ty cuối cùng để thực tập tại đó.

\paragraph*{Điều kiện cuối}

Nếu ca sử dụng thành công, trạng thái thực tập của người dùng đã được
thay đổi và hiển thị ngay trên trang thông tin thực tập.

\paragraph*{Các vấn đề mở}

Không có.

\end{document}