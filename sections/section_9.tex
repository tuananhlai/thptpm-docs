% !TeX root = ../main.tex
\documentclass[./../main.tex]{subfiles}

\begin{document}

Trong thời gian tới, hệ thống sẽ được triển khai thực tế để phục vụ mục
đích thực tập cho sinh viên trường Đại học Công Nghệ. Hệ thống hiện tại
đã cung cấp đầy đủ các tính năng cần thiết cho những người dùng chính
của hệ thống. Người dùng sinh viên, giảng viên và đối tác giờ đã có thể
hoàn thành các công việc chính của mình trên giao diện của hệ thống. Mặt
khác, người dùng quản trị viên đã được cung cấp bộ công cụ để lọc, xuất
và thống kê dữ liệu.

Yêu cầu phi chức năng của hệ thống cũng đã được đáp ứng. Bằng việc áp
dụng mô hình \emph{tích hợp liên tục / triển khai liên tục} (CI/CD) vào
quá trình phát triển phần mềm, độ ổn định của hệ thống khi chạy thực tế
đã được đảm bảo. Bên cạnh đó, các bản vá lỗi, cập nhật cũng có thể được
đưa lên trong một thời gian ngắn để tránh những sự cố nghiêm trọng.

Qua quá trình xây dựng hệ thống, em đã có cơ hội nghiên cứu và áp dụng
mô hình Microservice vào một dự án phần mềm thực tế. Em học được về ưu
điểm, nhược điểm và bộ công cụ đi kèm với kiến trúc này. Đồng thời, em
được biết thêm về cách triển khai một ứng dụng lên môi trường thực tế,
xin và duy trì chứng nhận TLS cho domain của mình.

Ngoài kiến thức chuyên môn, em đã học được thêm về những kỹ năng mềm
khác như làm việc nhóm, quản lý tiến độ, quản lý thời gian, phân công
công việc và các công cụ liên quan như Trello, Notion, \ldots{} Em cũng
được tham gia vào quá trình thu thập, phân tích yêu cầu và thiết kế ca
sử dụng. Đây là những kiến thức thật sự bổ ích, giúp em hiểu sâu hơn về
mọi pha trong một dự án phát triển phần mềm.

\end{document}