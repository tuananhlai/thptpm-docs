% !TeX root = ../main.tex
\documentclass[./../main.tex]{subfiles}

\begin{document}

\subsection{Quản lý bài đăng của mình}

\paragraph*{Mô tả tóm tắt}

Người dùng quản lý bài đăng của mình có thể xem, lọc, tìm kiếm, tạo mới, chỉnh sửa bài đăng.

\paragraph*{Luồng chính} Ca sử dụng bắt đầu khi người dùng mong muốn quản lý bài đăng.

\begin{table}[H]
	\caption{Quản lý bài đăng của mình - Luồng chính}
	\label{tab:partner_manage_posts}
	\begin{tabularx}{\textwidth}{|X|X|X|}
		\hline
		\textbf{Hành động}                                                                                  & \textbf{Hệ thống phản hồi}                                         & \textbf{Dữ liệu}       \\ \hline
		1. Người dùng truy cập trang Danh sách bài đăng.                                                    & 2. Hệ thống hiển thị danh sách bài đăng.                           &                        \\ \hline
		3. Người dùng thực hiện nhập từ khóa tìm kiếm hoặc lọc theo tùy chọn để tìm kiếm bài đăng mình cần. & 4. Hệ thống hiển thị danh sách bài đăng theo yêu cầu.              & Từ khóa, tùy chọn lọc. \\ \hline
		5. Người dùng yêu cầu tạo/sửa bài đăng.                                                             & 6. Hệ thống hiển thị giao diện phần thêm mới / chỉnh sửa bài đăng. &                        \\ \hline
		7. Người dùng nhập thông tin cho bài đăng và thực hiện tạo mới / chỉnh sửa.                         & 8. Hệ thống ghi nhận thông tin vào tạo mới / chỉnh sửa bài đăng.   & Thông tin bài đăng     \\ \hline
	\end{tabularx}
\end{table}

\paragraph*{Luồng thay thế} Không có.

\paragraph*{Luồng ngoại lệ}

\begin{itemize}
	\item

	      Tại bước 7: Người dùng nhập thiếu thông tin, hệ thống yêu cầu nhập đủ.

\end{itemize}

\paragraph*{Yêu cầu đặc biệt}

Người dùng là \textbf{đối tác}.

\paragraph*{Điều kiện đầu}

Người dùng đã đăng nhập thành công, và mong muốn thực hiện các thao tác quản lý bài đăng.

\paragraph*{Điều kiện cuối}

Nếu ca sử dụng thành công, người dùng đã tạo mới, chỉnh sửa, xem, lọc, tìm kiếm bài đăng thành công.

\paragraph*{Các vấn đề mở}

Không có.

\subsection{Quản lý yêu cầu thực tập của sinh viên}

\paragraph*{Mô tả tóm tắt}

Người dùng mong muốn xem, lọc, tìm kiếm, thực hiện chấp nhận hoặc từ chối yêu cầu thực tập của sinh viên.

\paragraph*{Luồng chính} Ca sử dụng bắt đầu khi người dùng mong muốn quản lý yêu cầu thực tập của sinh viên.

\begin{table}[H]
	\caption{Quản lý yêu cầu thực tập của sinh viên - Luồng chính}
	\label{tab:partner_manage_requests}
	\begin{tabularx}{\textwidth}{|X|X|X|}
		\hline
		\textbf{Hành động}                                                          & \textbf{Hệ thống phản hồi}                                                   & \textbf{Dữ liệu}              \\ \hline
		1. Người dùng truy cập trang Yêu cầu đăng ký thực tập.                      & 2. Hệ thống hiển thị danh sách yêu cầu của sinh viên.                        &                               \\ \hline
		3. Người dùng thực hiện nhập từ khóa tìm kiếm hoặc lọc theo tùy chọn.       & 4. Hệ thống hiển thị danh sách theo yêu cầu.                                 & Từ khóa, tùy chọn.            \\ \hline
		5. Người dùng thực hiện chấp nhận / từ chối yêu cầu của sinh viên.          & 6. Hệ thống ghi nhận yêu cầu, cập nhật lại trạng thái và phản hồi thông báo. & Thao tác chấp nhận / từ chối. \\ \hline
		7. Người dùng nhập thông tin cho bài đăng và thực hiện tạo mới / chỉnh sửa. & 8. Hệ thống ghi nhận thông tin vào tạo mới / chỉnh sửa bài đăng.             & Thông tin bài đăng            \\ \hline
	\end{tabularx}
\end{table}

\paragraph*{Luồng thay thế} Không có.

\paragraph*{Luồng ngoại lệ} Không có.

\paragraph*{Yêu cầu đặc biệt}

Người dùng là \textbf{đối tác}.

\paragraph*{Điều kiện đầu}

Người dùng đã đăng nhập thành công vào hệ thống và mong muốn thực hiện các thao tác quản lý yêu cầu thực tập.

\paragraph*{Điều kiện cuối}

Nếu ca sử dụng thành công, người dùng đã có thể xem, lọc, tìm kiếm, thực hiện chấp nhận hoặc từ chối yêu cầu của sinh viên.

\paragraph*{Các vấn đề mở}

Không có.

\subsection{Quản lý sinh viên đang thực tập tại công ty}

\paragraph*{Mô tả tóm tắt}

Người dùng mong muốn xem, lọc, tìm kiếm sinh viên đang thực tập tại công ty.

\paragraph*{Luồng chính} Ca sử dụng bắt đầu khi người dùng mong muốn quản lý sinh viên đang thực tập tại công ty.

\begin{table}[H]
	\caption{Quản lý sinh viên đang thực tập tại công ty - Luồng chính}
	\label{tab:partner_manage_students}
	\begin{tabularx}{\textwidth}{|X|X|X|}
		\hline
		\textbf{Hành động}                                                                          & \textbf{Hệ thống phản hồi}                                                   & \textbf{Dữ liệu}              \\ \hline
		1. Người dùng truy cập trang Sinh viên đang thực tập.                                       & 2. Hệ thống hiển thị danh sách sinh viên đang thực tập tại công ty.          &                               \\ \hline
		3. Người dùng thực hiện nhập từ khóa tìm kiếm hoặc lọc theo tùy chọn để tìm kiếm sinh viên. & 4. Hệ thống hiển thị danh sách sinh viên theo yêu cầu.                       & Từ khóa, tùy chọn.            \\ \hline
		5. Người dùng thực hiện chấp nhận / từ chối yêu cầu của sinh viên.                          & 6. Hệ thống ghi nhận yêu cầu, cập nhật lại trạng thái và phản hồi thông báo. & Thao tác chấp nhận / từ chối. \\ \hline
		7. Người dùng nhập thông tin cho bài đăng và thực hiện tạo mới / chỉnh sửa.                 & 8. Hệ thống ghi nhận thông tin vào tạo mới / chỉnh sửa bài đăng.             & Thông tin bài đăng            \\ \hline
	\end{tabularx}
\end{table}

\paragraph*{Luồng thay thế} Không có.

\paragraph*{Luồng ngoại lệ} Không có.

\paragraph*{Yêu cầu đặc biệt}

Người dùng là \textbf{đối tác}.

\paragraph*{Điều kiện đầu}

Người dùng đã đăng nhập thành công vào hệ thống và mong muốn thực hiện các thao tác quản lý sinh viên đang thực tập tại công ty.

\paragraph*{Điều kiện cuối}

Nếu ca sử dụng thành công, người dùng đã có thể xem, lọc và tìm kiếm sinh viên đang thực tập tại công ty mình.

\paragraph*{Các vấn đề mở}

Không có.

\subsection{Quản lý kỳ thực tập}

\paragraph*{Mô tả tóm tắt}

Người dùng mong muốn xem, tìm kiếm, tạo mới, chỉnh sửa danh sách kỳ thực tập.

\paragraph*{Luồng chính} Ca sử dụng bắt đầu khi người dùng mong muốn quản lý kỳ thực tập.

\begin{table}[H]
	\caption{Quản lý kỳ thực tập - Luồng chính}
	\label{tab:manage_terms}
	\begin{tabularx}{\textwidth}{|X|X|X|}
		\hline
		\textbf{Hành động}                                                   & \textbf{Hệ thống phản hồi}                                            & \textbf{Dữ liệu}       \\ \hline
		1. Người dùng truy cập trang Danh sách kỳ thực tập.                  & 2. Hệ thống hiển thị danh sách các kỳ thực tập.                       &                        \\ \hline
		3. Người dùng thao tác tìm kiếm kỳ thực tập theo năm.                & 4. Hệ thống hiển thị các kỳ thực tập của năm cần tìm.                 & Năm.                   \\ \hline
		5. Người dùng yêu cầu tạo mới / chỉnh sửa kỳ thực tập.               & 6. Hệ thống hiển thị giao diện phần thêm mới / chỉnh sửa kỳ thực tập. &                        \\ \hline
		7. Người dùng nhập và gửi thông tin tạo mới / chỉnh sửa kỳ thực tập. & 8. Hệ thống ghi nhận thông tin kỳ thực tập vừa tạo mới / chỉnh sửa.   & Thông tin kỳ thực tập. \\ \hline
	\end{tabularx}
\end{table}

\paragraph*{Luồng thay thế} Không có.

\paragraph*{Luồng ngoại lệ}

\begin{itemize}
	\item

	      Tại bước 3: Người dùng nhập năm không phù hợp, hệ thống không thay đổi.

	\item

	      Tại bước 7: Người dùng nhập thiếu thông tin hoặc thông tin không phù hợp, hệ thống yêu cầu nhập đủ.

\end{itemize}

\paragraph*{Yêu cầu đặc biệt}

Người dùng là \textbf{quản trị viên Khoa}.

\paragraph*{Điều kiện đầu}

Người dùng đã đăng nhập thành công và mong muốn thực hiện các thao tác quản lý danh sách kỳ thực tập.

\paragraph*{Điều kiện cuối}

Nếu ca sử dụng thành công, người dùng đã có thể xem, tìm kiếm, tạo mới, chỉnh sửa kỳ thực tập.

\paragraph*{Các vấn đề mở}

Không có.

\subsection{Quản lý danh sách sinh viên đang thực tập}

\paragraph*{Mô tả tóm tắt}

Người dùng mong muốn xem, lọc, tìm kiếm sinh viên, gán giảng viên cho sinh viên, tải lên/ tải xuống danh sách sinh viên.

\paragraph*{Luồng chính} Ca sử dụng bắt đầu khi người dùng mong muốn quản lý danh sách sinh viên đang thực tập.

\begin{table}[H]
	\caption{Quản lý danh sách sinh viên đang thực tập - Luồng chính}
	\label{tab:orgAdmin_manage_internship_students}
	\begin{tabularx}{\textwidth}{|X|X|X|}
		\hline
		\textbf{Hành động}                                                                                            & \textbf{Hệ thống phản hồi}                                                                      & \textbf{Dữ liệu}                            \\ \hline
		1. Người dùng truy cập trang Sinh viên trong Kỳ thực tập.                                                     & 2. Hệ thống hiển thị danh sách sinh viên trong kỳ thực tập.                                     &                                             \\ \hline
		3. Người dùng thực hiện nhập từ khóa tìm kiếm, lọc theo tùy chọn để tìm kiếm sinh viên.                       & 4. Hệ thống hiển thị danh sách sinh viên theo yêu cầu.                                          & Từ khóa, tùy chọn.                          \\ \hline
		5. Người dùng thực hiện gán giảng viên cho sinh viên bằng cách chọn và gửi danh sách sinh viên và giảng viên. & 8. Hệ thống ghi nhận yêu cầu và phản hồi thông báo.                                             & Giảng viên và danh sách sinh viên.          \\ \hline
		7. Người dùng thực hiện tải lên danh sách sinh viên.                                                          & 8. Hệ thống ghi nhận yêu cầu, phản hồi thông báo và cập nhật lại thông tin danh sách sinh viên. & File danh sách sinh viên trong kỳ thực tập. \\ \hline
		9. Người dùng thực hiện tải xuống danh sách sinh viên.                                                        & 10. Hệ thống ghi nhận yêu cầu và trả về cho người dùng file cần tải.                            &                                             \\ \hline
	\end{tabularx}
\end{table}

\paragraph*{Luồng thay thế} Không có.

\paragraph*{Luồng ngoại lệ}

\begin{itemize}
	\item

	      Tại bước 5: Người dùng không chọn sinh viên hoặc giảng viên, hệ thống phản hồi lại yêu cầu chọn lại.

\end{itemize}

\paragraph*{Yêu cầu đặc biệt}

Người dùng là \textbf{quản trị viên Khoa}.

\paragraph*{Điều kiện đầu}

Người dùng đã đăng nhập thành công vào hệ thống và mong muốn thực hiện các thao tác quản lý danh sách sinh viên trong kỳ thực tập.

\paragraph*{Điều kiện cuối}

Nếu ca sử dụng thành công, người dùng có thể xem, lọc, tìm kiếm danh sách sinh viên, gán giảng viên cho sinh viên, tải lên / tải xuống danh sách sinh viên trong kỳ thực tập.

\paragraph*{Các vấn đề mở}

Không có.

\subsection{Quản lý danh sách đối tác trong kỳ thực tập}

\paragraph*{Mô tả tóm tắt}

Người dùng mong muốn thực hiện xem, lọc, tìm kiếm danh sách đối tác, chấp nhận / từ chối công ty.

\paragraph*{Luồng chính} Ca sử dụng bắt đầu khi người dùng mong muốn quản lý danh sách danh sách đối tác trong kỳ thực tập.

\begin{table}[H]
	\caption{Quản lý danh sách đối tác trong kỳ thực tập - Luồng chính}
	\label{tab:orgAdmin_manage_internship_partners}
	\begin{tabularx}{\textwidth}{|X|X|X|}
		\hline
		\textbf{Hành động}                                                                            & \textbf{Hệ thống phản hồi}                                                                      & \textbf{Dữ liệu}                            \\ \hline
		1. Người dùng truy cập trang Đơn vị thực tập trong Kỳ thực tập.                               & 2. Hệ thống hiển thị danh sách đơn vị thực tập trong kỳ thực tập.                               &                                             \\ \hline
		3. Người dùng thực hiện nhập từ khóa tìm kiếm, lọc theo tùy chọn để tìm kiếm đơn vị thực tập. & 4. Hệ thống hiển thị danh sách đơn vị thực tập theo yêu cầu.                                    & Từ khóa, tùy chọn lọc.                      \\ \hline
		5. Người dùng thực hiện chấp nhận / từ chối công ty.                                          & 6. Hệ thống ghi nhận yêu cầu, phản hồi thông báo và cập nhật danh sách đơn vị thực tập.         &                                             \\ \hline
		7. Người dùng thực hiện tải lên danh sách sinh viên.                                          & 8. Hệ thống ghi nhận yêu cầu, phản hồi thông báo và cập nhật lại thông tin danh sách sinh viên. & File danh sách sinh viên trong kỳ thực tập. \\ \hline
		9. Người dùng thực hiện tải xuống danh sách sinh viên.                                        & 10. Hệ thống ghi nhận yêu cầu và trả về cho người dùng file cần tải.                            &                                             \\ \hline
	\end{tabularx}
\end{table}

\paragraph*{Luồng thay thế} Không có.

\paragraph*{Luồng ngoại lệ} Không có.

\paragraph*{Yêu cầu đặc biệt}

Người dùng là \textbf{quản trị viên Khoa}.

\paragraph*{Điều kiện đầu}

Người dùng đã đăng nhập thành công và mong muốn thực hiện các thao tác quản lý đơn vị thực tập trong kỳ thực tập.

\paragraph*{Điều kiện cuối}

Nếu ca sử dụng thành công, người dùng đã có thể thực hiện xem, lọc, tìm kiếm danh sách đơn vị thực tập, chấp nhận / từ chối công ty.

\paragraph*{Các vấn đề mở}

Không có.

\subsection{Quản lý bài đăng của các đối tác trong kỳ thực tập}

\paragraph*{Mô tả tóm tắt}

Người dùng mong muốn xem, tìm kiếm, thêm / sửa bài đăng của các đối tác.

\paragraph*{Luồng chính} Ca sử dụng bắt đầu khi người dùng mong muốn quản lý bài đăng của tất cả đối tác trong kỳ thực tập.

\begin{table}[H]
	\caption{Quản lý bài đăng của các đối tác trong kỳ thực tập - Luồng chính}
	\label{tab:orgAdmin_manage_internship_posts}
	\begin{tabularx}{\textwidth}{|X|X|X|}
		\hline
		\textbf{Hành động}                                       & \textbf{Hệ thống phản hồi}                                                                              & \textbf{Dữ liệu}                            \\ \hline
		1. Người dùng truy cập trang Bài đăng trong kỳ thực tập. & 2. Hệ thống hiển thị danh sách bài đăng của các đối tác.                                                &                                             \\ \hline
		3. Người dùng thực hiện tìm kiếm bài đăng bằng từ khóa.  & 4. Hệ thống hiển thị danh sách bài đăng theo yêu cầu.                                                   & Từ khóa.                                    \\ \hline
		5. Người dùng thực hiện tạo mới / chỉnh sửa bài đăng.    & 6. Hệ thống ghi nhận thông tin gửi lên, lưu lại, cập nhật lại danh sách bài đăng và phản hồi thông báo. & Thông tin bài đăng.                         \\ \hline
		7. Người dùng thực hiện tải lên danh sách sinh viên.     & 8. Hệ thống ghi nhận yêu cầu, phản hồi thông báo và cập nhật lại thông tin danh sách sinh viên.         & File danh sách sinh viên trong kỳ thực tập. \\ \hline
		9. Người dùng thực hiện tải xuống danh sách sinh viên.   & 10. Hệ thống ghi nhận yêu cầu và trả về cho người dùng file cần tải.                                    &                                             \\ \hline
	\end{tabularx}
\end{table}

\paragraph*{Luồng thay thế} Không có.

\paragraph*{Luồng ngoại lệ}

\begin{itemize}
	\item

	      Tại bước 5: Người dùng nhập thiếu thông tin bài đăng, hệ thống yêu cầu nhập đủ.

\end{itemize}

\paragraph*{Yêu cầu đặc biệt}

Người dùng là \textbf{quản trị viên Khoa}.

\paragraph*{Điều kiện đầu}

Người dùng đã đăng nhập thành công vào hệ thống và mong muốn thực hiện các thao tác quản lý danh sách bài đăng của các đối tác.

\paragraph*{Điều kiện cuối}

Nếu ca sử dụng thành công, người dùng đã có thể xem, tìm kiếm, tạo mới / chỉnh sửa bài đăng của đối tác.

\paragraph*{Các vấn đề mở}

Không có.

\subsection{Quản lý giảng viên hướng dẫn trong kỳ thực tập}

\paragraph*{Mô tả tóm tắt}

Người dùng mong muốn xem, tìm kiếm giảng viên trong kỳ thực tập.

\paragraph*{Luồng chính} Ca sử dụng bắt đầu khi người dùng mong muốn quản lý giảng viên hướng dẫn trong kỳ thực tập.

\begin{table}[H]
	\caption{Quản lý giảng viên hướng dẫn trong kỳ thực tập - Luồng chính}
	\label{tab:orgAdmin_manage_internship_lecturers}
	\begin{tabularx}{\textwidth}{|X|X|X|}
		\hline
		\textbf{Hành động}                                             & \textbf{Hệ thống phản hồi}                              & \textbf{Dữ liệu} \\ \hline
		1. Người dùng truy cập vào trang Giảng viên trong kỳ thực tập. & 2. Hệ thống hiển thị danh sách giảng viên trong kỳ.     &                  \\ \hline
		3. Người dùng thực hiện tìm kiếm giảng viên theo từ khóa.      & 4. Hệ thống hiển thị danh sách giảng viên theo yêu cầu. & Từ khóa.         \\ \hline
	\end{tabularx}
\end{table}

\paragraph*{Luồng thay thế} Không có.

\paragraph*{Luồng ngoại lệ} Không có.

\paragraph*{Yêu cầu đặc biệt}

Người dùng là \textbf{quản trị viên Khoa}.

\paragraph*{Điều kiện đầu}

Người dùng đã đăng nhập thành công vào hệ thống và mong muốn thực hiện các thao tác quản lý giảng viên trong kỳ thực tập.

\paragraph*{Điều kiện cuối}

Nếu ca sử dụng thành công, người dùng đã có thể xem, tìm kiếm danh sách giảng viên theo kỳ thực tập.

\paragraph*{Các vấn đề mở}

Không có.

\subsection{Quản lý danh sách sinh viên}

\paragraph*{Mô tả tóm tắt}

Người dùng mong muốn xem, lọc, tìm kiếm thông tin cá nhân của sinh viên.

\paragraph*{Luồng chính} Ca sử dụng bắt đầu khi người dùng mong muốn quản lý danh sách sinh viên.

\begin{table}[H]
	\caption{Quản lý danh sách sinh viên - Luồng chính}
	\label{tab:orgAdmin_manage_students}
	\begin{tabularx}{\textwidth}{|X|X|X|}
		\hline
		\textbf{Hành động}                                                  & \textbf{Hệ thống phản hồi}                                      & \textbf{Dữ liệu}       \\ \hline
		1. Người dùng truy cập trang danh sách thông tin cá nhân Sinh viên. & 2. Hệ thống hiển thị danh sách thông tin cá nhân của sinh viên. &                        \\ \hline
		3. Người dùng thực hiện nhập từ khóa, lọc sinh viên theo lớp.       & 4. Hệ thống hiển thị danh sách sinh viên theo yêu cầu.          & Từ khóa, tùy chọn lọc. \\ \hline
	\end{tabularx}
\end{table}

\paragraph*{Luồng thay thế} Không có.

\paragraph*{Luồng ngoại lệ} Không có.

\paragraph*{Yêu cầu đặc biệt}

Người dùng là \textbf{quản trị viên Khoa, quản trị viên hệ thống}.

\paragraph*{Điều kiện đầu}

Người dùng đã đăng nhập thành công vào hệ thống và mong muốn thực hiện các thao tác quản lý danh sách thông tin cá nhân sinh viên.

\paragraph*{Điều kiện cuối}

Nếu ca sử dụng thành công, người dùng đã có thể xem, lọc, tìm kiếm thông tin cá nhân của sinh viên.

\paragraph*{Các vấn đề mở}

Không có.

\subsection{Quản lý danh sách đối tác}

\paragraph*{Mô tả tóm tắt}

Người dùng mong muốn xem, lọc, tìm kiếm thông tin đối tác, xem / thêm / sửa liên hệ cho đối tác.

\paragraph*{Luồng chính} Ca sử dụng bắt đầu khi người dùng mong muốn quản lý danh sách đối tác.

\begin{table}[H]
	\caption{Quản lý danh sách đối tác - Luồng chính}
	\label{tab:orgAdmin_manage_partners}
	\begin{tabularx}{\textwidth}{|X|X|X|}
		\hline
		\textbf{Hành động}                                                          & \textbf{Hệ thống phản hồi}                                                                      & \textbf{Dữ liệu}       \\ \hline
		1. Người dùng truy cập trang danh sách thông tin cá nhân Đối tác.           & 2. Hệ thống hiển thị danh sách đối tác.                                                         &                        \\ \hline
		3. Người dùng nhập vào từ khóa, hoặc lọc theo tùy chọn để tìm kiếm đối tác. & 4. Hệ thống hiển thị danh sách đối tác theo yêu cầu.                                            & Từ khóa, tùy chọn lọc. \\ \hline
		5. Người dùng yêu cầu xem danh sách liên hệ của đối tác.                    & 6. Hệ thống hiển thị danh sách liên hệ của đối tác.                                             &                        \\ \hline
		7. Người dùng thực hiện thêm mới / chỉnh sửa liên hệ.                       & 8. Hệ thống hiển thị phần giao diện thêm mới / chỉnh sửa liên hệ.                               &                        \\ \hline
		9. Người dùng nhập vào thông tin bài đăng cần tạo mới / chỉnh sửa.          & 10. Hệ thống ghi nhận thông tin, lưu lại, cập nhật lại danh sách liên hệ và phản hồi thông báo. & Thông tin liên hệ.     \\ \hline
	\end{tabularx}
\end{table}

\paragraph*{Luồng thay thế} Không có.

\paragraph*{Luồng ngoại lệ}

\begin{itemize}
	\item

	      Tại bước 9: Người dùng nhập thông tin không đúng yêu cầu, hệ thống yêu cầu nhập lại.

\end{itemize}

\paragraph*{Yêu cầu đặc biệt}

Người dùng là \textbf{quản trị viên Khoa, quản trị viên hệ thống}.

\paragraph*{Điều kiện đầu}

Người dùng đã đăng nhập thành công vào hệ thống và mong muốn thực hiện các thao tác quản lý thông tin đối tác.

\paragraph*{Điều kiện cuối}

Nếu ca sử dụng thành công, người dùng đã có thể xem, lọc, tìm kiếm thông tin đối tác, xem / tạo mới / chỉnh sửa liên hệ cho đối tác.

\paragraph*{Các vấn đề mở}

Không có.

\subsection{Quản lý danh sách giảng viên}

\paragraph*{Mô tả tóm tắt}

Người dùng mong muốn xem, tìm kiếm thông tin cá nhân của giảng viên.

\paragraph*{Luồng chính} Ca sử dụng bắt đầu khi người dùng mong muốn quản lý danh sách giảng viên.

\begin{table}[H]
	\caption{Quản lý danh sách giảng viên - Luồng chính}
	\label{tab:orgAdmin_manage_lecturers}
	\begin{tabularx}{\textwidth}{|X|X|X|}
		\hline
		\textbf{Hành động}                                                       & \textbf{Hệ thống phản hồi}                                       & \textbf{Dữ liệu} \\ \hline
		1. Người dùng truy cập trang danh sách thông tin cá nhân của giảng viên. & 2. Hệ thống hiển thị danh sách thông tin cá nhân của giảng viên. &                  \\ \hline
		3. Người dùng thực hiện tìm kiếm giảng viên theo từ khóa.                & 4. Hệ thống hiển thị danh sách giảng viên theo yêu cầu.          & Từ khóa.         \\ \hline
	\end{tabularx}
\end{table}

\paragraph*{Luồng thay thế} Không có.

\paragraph*{Luồng ngoại lệ} Không có.

\paragraph*{Yêu cầu đặc biệt}

Người dùng là \textbf{quản trị viên Khoa, quản trị viên hệ thống}.

\paragraph*{Điều kiện đầu}

Người dùng đã đăng nhập thành công vào hệ thống và mong muốn thực hiện các thao tác quản lý danh sách thông tin cá nhân của giảng viên.

\paragraph*{Điều kiện cuối}

Nếu ca sử dụng thành công, người dùng đã có thể xem, tìm kiếm danh sách thông tin cá nhân của giảng viên.

\paragraph*{Các vấn đề mở}

Không có.

\subsection{Quản lý danh sách người dùng}

\paragraph*{Mô tả tóm tắt}

Người dùng mong muốn xem, lọc, tìm kiếm người dùng.

\paragraph*{Luồng chính} Ca sử dụng bắt đầu khi người dùng mong muốn quản lý danh sách người dùng.

\begin{table}[H]
	\caption{Quản lý danh sách người dùng - Luồng chính}
	\label{tab:orgAdmin_manage_users}
	\begin{tabularx}{\textwidth}{|X|X|X|}
		\hline
		\textbf{Hành động}                                                             & \textbf{Hệ thống phản hồi}                                & \textbf{Dữ liệu}       \\ \hline
		1. Người dùng truy cập trang Danh sách người dùng.                             & 2. Hệ thống hiển thị danh sách người dùng trong hệ thống. &                        \\ \hline
		3. Người dùng thực hiện lọc, tìm kiếm thông tin danh sách người dùng hệ thống. & 4. Hệ thống hiển thị danh sách người dùng theo yêu cầu.   & Từ khóa, tùy chọn lọc. \\ \hline
	\end{tabularx}
\end{table}

\paragraph*{Luồng thay thế} Không có.

\paragraph*{Luồng ngoại lệ} Không có.

\paragraph*{Yêu cầu đặc biệt}

Người dùng là \textbf{quản trị viên hệ thống}.

\paragraph*{Điều kiện đầu}

Người dùng đã đăng nhập thành công và mong muốn thực hiện các thao tác quản lý danh sách người dùng.

\paragraph*{Điều kiện cuối}

Nếu ca sử dụng thành công, người dùng đã có thể xem, lọc, tìm kiếm danh sách người dùng.

\paragraph*{Các vấn đề mở}

Không có.

\subsection{Quản lý danh sách khoa}

\paragraph*{Mô tả tóm tắt}

Người dùng mong muốn xem, tìm kiếm, thêm mới / chỉnh sửa thông tin một Khoa.

\paragraph*{Luồng chính} Ca sử dụng bắt đầu khi người dùng mong muốn quản lý danh sách khoa.

\begin{table}[H]
	\caption{Quản lý danh sách khoa - Luồng chính}
	\label{tab:admin_manage_orgs}
	\begin{tabularx}{\textwidth}{|X|X|X|}
		\hline
		\textbf{Hành động}                                                 & \textbf{Hệ thống phản hồi}                                                                      & \textbf{Dữ liệu}   \\ \hline
		1. Người dùng truy cập danh sách Khoa.                             & 2. Hệ thống hiển thị danh sách Khoa.                                                            &                    \\ \hline
		3. Người dùng thực hiện tìm kiếm Khoa theo từ khóa.                & 4. Hệ thống hiển thị danh sách Khoa theo yêu cầu.                                               & Từ khóa.           \\ \hline
		5. Người dùng yêu cầu tạo mới / chỉnh sửa một Khoa.                & 6. Hệ thống hiển thị giao diện tạo mới / chỉnh sửa một Khoa.                                    &                    \\ \hline
		7. Người dùng nhập thông tin của Khoa cần tạo mới / chỉnh sửa.     & 8. Hệ thống ghi nhận thông tin, lưu lại, cập nhật lại danh sách các Khoa và phản hồi thông báo. & Thông tin Khoa.    \\ \hline
		9. Người dùng nhập vào thông tin bài đăng cần tạo mới / chỉnh sửa. & 10. Hệ thống ghi nhận thông tin, lưu lại, cập nhật lại danh sách liên hệ và phản hồi thông báo. & Thông tin liên hệ. \\ \hline
	\end{tabularx}
\end{table}

\paragraph*{Luồng thay thế} Không có.

\paragraph*{Luồng ngoại lệ}

\begin{itemize}
	\item

	      Tại bước 7: Người dùng nhập sai / thiếu thông tin, hệ thống yêu cầu nhập lại

\end{itemize}

\paragraph*{Yêu cầu đặc biệt}

Người dùng là \textbf{quản trị viên hệ thống}.

\paragraph*{Điều kiện đầu}

Người dùng đã đăng nhập hệ thống thành công và mong muốn thực hiện các thao tác quản lý danh sách Khoa.

\paragraph*{Điều kiện cuối}

Nếu ca sử dụng thành công, người dùng đã có thể xem, tìm kiếm, tạo mới / chỉnh sửa thông tin Khoa.

\paragraph*{Các vấn đề mở}

Không có.

\subsection{Quản lý danh sách lớp}

\paragraph*{Mô tả tóm tắt}

Người dùng mong muốn xem, tìm kiếm, tạo mới / chỉnh sửa thông tin lớp học.

\paragraph*{Luồng chính} Ca sử dụng bắt đầu khi người dùng mong muốn quản lý danh sách lớp.

\begin{table}[H]
	\caption{Quản lý danh sách lớp - Luồng chính}
	\label{tab:admin_manage_classes}
	\begin{tabularx}{\textwidth}{|X|X|X|}
		\hline
		\textbf{Hành động}                                            & \textbf{Hệ thống phản hồi}                                                                     & \textbf{Dữ liệu} \\ \hline
		1. Người dùng truy cập danh sách lớp.                         & 2. Hệ thống hiển thị danh sách lớp.                                                            &                  \\ \hline
		3. Người dùng thực hiện tìm kiếm lớp theo từ khóa.            & 4. Hệ thống hiển thị danh sách lớp theo yêu cầu.                                               & Từ khóa.         \\ \hline
		5. Người dùng yêu cầu tạo mới / chỉnh sửa một lớp.            & 6. Hệ thống hiển thị giao diện tạo mới / chỉnh sửa một lớp.                                    &                  \\ \hline
		7. Người dùng nhập thông tin của lớp cần tạo mới / chỉnh sửa. & 8. Hệ thống ghi nhận thông tin, lưu lại, cập nhật lại danh sách các lớp và phản hồi thông báo. & Thông tin lớp.   \\ \hline
	\end{tabularx}
\end{table}

\paragraph*{Luồng thay thế} Không có.

\paragraph*{Luồng ngoại lệ}

\begin{itemize}
	\item

	      Tại bước 7: Tại bước 7: Người dùng nhập sai / thiếu thông tin, hệ thống yêu cầu nhập lại.

\end{itemize}

\paragraph*{Yêu cầu đặc biệt}

Người dùng là \textbf{quản trị viên hệ thống}.

\paragraph*{Điều kiện đầu}

Người dùng đã đăng nhập hệ thống thành công và mong muốn thực hiện các thao tác quản lý danh sách lớp.

\paragraph*{Điều kiện cuối}

Nếu ca sử dụng thành công, người dùng đã có thể xem, tìm kiếm, tạo mới / chỉnh sửa thông tin lớp.

\paragraph*{Các vấn đề mở}

Không có.

\subsection{Các ca sử dụng khác}

Ngoài các ca sử dụng nêu trên, còn có các ca sử dụng khác của hệ thống được trình bày ở khóa luận của bạn Lại Tuấn Anh:

\begin{itemize}
	\item

	      Đăng nhập hệ thống

	\item

	      Đặt lại mật khẩu

	\item

	      Đổi mật khẩu

	\item

	      Chỉnh sửa thông tin cá nhân

	\item

	      Xem thông tin bài đăng

	\item

	      Đăng ký thực tập với công ty liên kết

	\item

	      Đăng ký thực tập với công ty ngoài

	\item

	      Đăng ký phỏng vấn từ bài đăng

	\item

	      Xem thông tin thực tập

	\item

	      Chọn công ty để thực tập

	\item

	      Nộp báo cáo thực tập

	\item

	      Tải lên CV

	\item

	      Xem thống kê dữ liệu trên hệ thống

	\item

	      Xem danh sách sinh viên đang hướng dẫn

	\item

	      Xem báo cáo thực tập của sinh viên

	\item

	      Chấm điểm cho sinh viên
\end{itemize}

\end{document}