% !TeX root = ../main.tex
\documentclass[./../main.tex]{subfiles}

\begin{document}

\subsection{Đăng nhập hệ thống}

\paragraph*{Mô tả tóm tắt}

Người dùng đăng nhập vào hệ thống với tài khoản được cấp.

\paragraph*{Luồng chính} Ca sử dụng này bắt đầu khi người dùng mong muốn đăng
nhập vào hệ thống.

\begin{table}[H]
	\caption{Đăng nhập hệ thống - luồng chính}
	\label{tab:login_flow_main}
	\begin{tabularx}{\textwidth}{|X|X|X|}
		\hline
		\textbf{Hành động}                      & \textbf{Hệ thống phản hồi}                                                       & \textbf{Dữ liệu}              \\ \hline
		1. Người dùng truy cập trang đăng nhập. & 2. Hệ thống yêu cầu người dùng nhập email và mật khẩu.                           &                               \\ \hline
		3. Người dùng điền các thông tin.       & 4. Hệ thống kiểm định thông tin và điều hướng người dùng vào hệ thống theo Role. & Email và mật khẩu người dùng. \\ \hline
	\end{tabularx}
\end{table}

\paragraph*{Luồng thay thế}

\begin{itemize}
	\item
	      Tại bước 3: Thay vì đăng nhập bằng email và mật khẩu, người dùng yêu
	      cầu đăng nhập bằng Google, hệ thống điều hướng và mở một trang đăng
	      nhập bằng tài khoản Google.
\end{itemize}

\paragraph*{Luồng ngoại lệ}

\begin{itemize}
	\item
	      Tại bước 3: Người dùng điền thiếu hoặc sai thông tin, hệ thống yêu cầu
	      nhập lại.
	\item
	      Tại bước 4: Hệ thống kiểm định thông tin đăng nhập sai, hệ thống hiển
	      thị thông tin lỗi, yêu cầu nhập lại. Người dùng có thể tiếp tục lặp
	      lại luồng sự kiện hoặc thoát khỏi trang đăng nhập.
\end{itemize}

\paragraph*{Yêu cầu đặc biệt}

Người dùng đã có tài khoản trong hệ thống.

\paragraph*{Điều kiện đầu}

Không có.

\paragraph*{Điều kiện cuối}

Nếu ca sử dụng thành công, người dùng đã được đăng nhập vào hệ thống,
ngược lại, trạng thái hệ thống không thay đổi.

\paragraph*{Các vấn đề mở}

Không có.

\subsection{Đặt lại mật khẩu}
\paragraph*{Mô tả tóm tắt}

Người dùng thực hiện đặt lại mật khẩu.


\paragraph*{Luồng chính} Ca sử dụng bắt đầu khi người dùng muốn đăng nhập vào
hệ thống nhưng quên mật khẩu.

\begin{table}[H]
	\caption{Đặt lại mật khẩu - Luồng chính}
	\label{tab:reset_password}
	\begin{tabularx}{\textwidth}{|X|X|X|}
		\hline
		\textbf{Hành động}                     & \textbf{Hệ thống phản hồi}                                                               & \textbf{Dữ liệu} \\ \hline
		1. Người dùng yêu cầu đặt lại mật khẩu & 2. Hệ thống yêu cầu người dùng nhập email.                                               &                  \\ \hline
		3. Người dùng nhập email.              & 4. Hệ thống kiểm định sự tồn tại của email và gửi đường link đặt lại mật khẩu qua email. & Email            \\ \hline
	\end{tabularx}
\end{table}

\paragraph*{Luồng thay thế} Không có

\paragraph*{Luồng ngoại lệ}

\begin{itemize}
	\item

	      Tại bước 3: Người dùng nhập vào email không tồn tại trong hệ thống. Hệ
	      thống yêu cầu nhập lại.

	\item

	      Tại bước 4: Hệ thống kiểm định email không tồn tại, hệ thống hiển thị
	      thông báo cho người dùng, yêu cầu nhập lại. Người dùng có thể tiếp tục
	      lặp lại luồng sự kiện hoặc thoát khỏi trang đăng nhập.

\end{itemize}

\paragraph*{Yêu cầu đặc biệt}

Người dùng đã có tài khoản trong hệ thống.

\paragraph*{Điều kiện đầu}

Không có.

\paragraph*{Điều kiện cuối}

Nếu ca sử dụng thành công, người dùng có thể thay đổi mật khẩu mới,
ngược lại, trạng thái hệ thống không thay đổi.

\paragraph*{Các vấn đề mở}

Không có.

\subsection{Đổi mật khẩu}

\paragraph*{Mô tả tóm tắt}

Người dùng thực hiện thay đổi mật khẩu tài khoản cá nhân.


\paragraph*{Luồng chính} Ca sử dụng bắt đầu khi người dùng đã đăng nhập thành công và muốn thay đổi mật khẩu.

\begin{table}[H]
	\caption{Đổi mật khẩu - Luồng chính}
	\label{tab:change_password}
	\begin{tabularx}{\textwidth}{|X|X|X|}
		\hline
		\textbf{Hành động}                                  & \textbf{Hệ thống phản hồi}                                                                              & \textbf{Dữ liệu}             \\ \hline
		1. Người dùng yêu cầu đổi mật khẩu.                 & 2. Hệ thống yêu cầu người dùng nhập mật khẩu cũ và mật khẩu mới.                                        &                              \\ \hline
		3. Người dùng nhập vào mật khẩu cũ và mật khẩu mới. & 4. Hệ thống kiểm định lại mật khẩu cũ và sự phù hợp của mật khẩu mới, rồi thông báo lại cho người dùng. & Mật khẩu cũ và mật khẩu mới. \\ \hline
	\end{tabularx}
\end{table}

\paragraph*{Luồng thay thế} Không có

\paragraph*{Luồng ngoại lệ}

\begin{itemize}
	\item

	      Tại bước 3: Người dùng nhập sai / thiếu thông tin hoặc mật khẩu mới
	      không đáp ứng yêu cầu, hệ thống yêu cầu nhập lại.

	\item

	      Tại bước 4: Hệ thống kiểm định thông tin sai / thiếu hoặc mật khẩu mới
	      không đáp ứng yêu cầu, hệ thống yêu cầu nhập lại. Người dùng có thể
	      tiếp tục lặp lại luồng sự kiện hoặc thoát khỏi trang hiện tại.

\end{itemize}

\paragraph*{Yêu cầu đặc biệt}

Không có.

\paragraph*{Điều kiện đầu}

Người dùng đã đăng nhập và hệ thống và yêu cầu thay đổi mật khẩu.

\paragraph*{Điều kiện cuối}

Nếu ca sử dụng thành công, người dùng đã thay đổi được mật khẩu, ngược
lại, trạng thái hệ thống không thay đổi.

\paragraph*{Các vấn đề mở}

Không có.

\subsection{Chỉnh sửa thông tin cá nhân}

\paragraph*{Mô tả tóm tắt}

Người dùng yêu cầu thay đổi các thông tin cá nhân.

\paragraph*{Luồng chính} Ca sử dụng bắt đầu khi người dùng đã đăng nhập thành
công và muốn thay đổi thông tin cá nhân.

\begin{table}[H]
	\caption{Chỉnh sửa thông tin cá nhân - Luồng chính}
	\label{tab:update_info}
	\begin{tabularx}{\textwidth}{|X|X|X|}
		\hline
		\textbf{Hành động}                                                          & \textbf{Hệ thống phản hồi}                                      & \textbf{Dữ liệu}                \\ \hline
		1. Người dùng yêu cầu thay đổi thông tin cá nhân.                           & 2. Hệ thống yêu cầu người dùng nhập các thông tin cần thay đổi. &                                 \\ \hline
		3. Người dùng nhập vào các thông tin cần thay đổi tại các trường tương ứng. & 4. Hệ thống phản hồi lại kết quả thay đổi thông tin cá nhân.    & Thông tin cá nhân cần thay đổi. \\ \hline
	\end{tabularx}
\end{table}

\paragraph*{Luồng thay thế} Không có.

\paragraph*{Luồng ngoại lệ} Không có.

\paragraph*{Yêu cầu đặc biệt}

Người dùng hệ thống là \textbf{sinh viên, giảng viên, đối tác.}

\paragraph*{Điều kiện đầu}

Người dùng đã đăng nhập hệ thống và mong muốn thay đổi thông tin cá
nhân.

\paragraph*{Điều kiện cuối}

Nếu ca sử dụng thành công, người dùng đã thay đổi được thông tin cá
nhân.

\paragraph*{Các vấn đề mở}

Không có

\subsection{Xem thông tin bài đăng}

\paragraph*{Mô tả tóm tắt}

Người dùng yêu cầu xem thông tin chi tiết bài đăng tuyển dụng.

\paragraph*{Luồng chính} Ca sử dụng bắt đầu khi người dùng đã đăng nhập vào
hệ thống và muốn xem thông tin bài đăng.

\begin{table}[H]
	\caption{Xem thông tin bài đăng - Luồng chính}
	\label{tab:read_post}
	\begin{tabularx}{\textwidth}{|X|X|X|}
		\hline
		\textbf{Hành động}                            & \textbf{Hệ thống phản hồi}                        & \textbf{Dữ liệu} \\ \hline
		1. Người dùng yêu cầu xem thông tin bài đăng. & 2. Hệ thống hiển thị thông tin chi tiết bài đăng. &                  \\ \hline
	\end{tabularx}
\end{table}

\paragraph*{Luồng thay thế} Không có.

\paragraph*{Luồng ngoại lệ} Không có.

\paragraph*{Yêu cầu đặc biệt}

Người dùng là \textbf{sinh viên}.

\paragraph*{Điều kiện đầu}

Người dùng đã đăng nhập vào hệ thống và mong muốn xem thông tin bài đăng
tuyển dụng.

\paragraph*{Điều kiện cuối}

Nếu ca sử dụng thành công, người dùng xem được thông tin có trong bài
đăng.

\paragraph*{Các vấn đề mở}

Không có.

% \subsection{Xem quy chế thực tập}

% \paragraph*{Mô tả tóm tắt}

% Người dùng lần đầu tiên đăng nhập vào hệ thống với tài khoản được cấp,
% một email có nội dung về quy chế thực tập sẽ được gửi đến email người
% dùng.


% \paragraph*{Luồng chính}

% \paragraph*{Luồng thay thế} Không có

% \paragraph*{Luồng ngoại lệ}

% \begin{itemize}
% \item

%   Tại bước 1: Người dùng nhập sai / thiếu thông tin, hệ thống yêu cầu
%   nhập lại.

% \item

%   Tại bước 2: Hệ thống kiểm định thấy thông tin sai, hệ thống không gửi
%   email, yêu cầu nhập lại. Người dùng có thể tiếp tục lặp lại luồng sự
%   kiện hoặc thoát khỏi trang hiện tại.

% \end{itemize}

% \paragraph*{Yêu cầu đặc biệt}

% Người dùng là \textbf{sinh viên} và lần đăng nhập này là đầu tiên.

% \paragraph*{Điều kiện đầu}

% Người dùng đăng nhập thành công vào hệ thống.

% \paragraph*{Điều kiện cuối}

% Nếu ca sử dụng thành công, người dùng sẽ nhận được email chứa nội dung
% quy chế thực tập.

% \paragraph*{Các vấn đề mở}

% Không có.

\subsection{Đăng ký thực tập với công ty liên kết}

\paragraph*{Mô tả tóm tắt}

Người dùng đã có nơi thực tập và là công ty đã liên kết với Khoa, mong
muốn đăng ký thực tập với công ty đó.

\paragraph*{Luồng chính} Ca sử dụng bắt đầu khi sinh viên muốn đăng ký thực tập với công ty đã liên kết với Khoa.

\begin{table}[H]
	\caption{Đăng ký thực tập với công ty liên kết - Luồng chính}
	\label{tab:register_company}
	\begin{tabularx}{\textwidth}{|X|X|X|}
		\hline
		\textbf{Hành động}                                                         & \textbf{Hệ thống phản hồi}                                                           & \textbf{Dữ liệu} \\ \hline
		1. Người dùng truy cập trang chủ của hệ thống.                             & 2. Hệ thống hiển thị giao diện trang chủ có phần tìm kiếm tên công ty đang thực tập. &                  \\ \hline
		3. Người dùng nhập tên công ty và chọn công ty có trong danh sách hiện ra. & 4. Hệ thống nhận thông tin và phản hồi lại thông báo.                                & Công ty đã chọn. \\ \hline
	\end{tabularx}
\end{table}

\paragraph*{Luồng thay thế} Không có.

\paragraph*{Luồng ngoại lệ}

\begin{itemize}
	\item
	      Tại bước 3: Người dùng chọn công ty đang ở trạng thái bị từ chối, hệ
	      thống phản hồi thông báo.
	\item

	      Tại bước 4: Hệ thống kiểm định thấy công ty đã chọn đang ở trạng thái
	      bị từ chối, gửi lại thông báo cho người dùng. Người dùng có thể tiếp
	      tục lặp lại luồng sự kiện hoặc thoát khỏi trang hiện tại.

\end{itemize}

\paragraph*{Yêu cầu đặc biệt}

Người dùng là \textbf{sinh viên}.

\paragraph*{Điều kiện đầu}

Người dùng đã đăng nhập thành công và mong muốn đăng ký thực tập tại
công ty đã liên kết.

\paragraph*{Điều kiện cuối}

Nếu ca sử dụng thành công, người dùng đã đăng ký thực tập thành công tại
công ty đã chọn, ngược lại, trạng thái hệ thống không thay đổi.

\paragraph*{Các vấn đề mở}

Không có.

\subsection{Đăng ký thực tập với công ty ngoài}

\paragraph*{Mô tả tóm tắt}

Người dùng đã có nơi thực tập và là công ty chưa liên kết với Khoa, mong
muốn đăng ký thực tập tại công ty đó.

\paragraph*{Luồng chính} Ca sử dụng bắt đầu khi người dùng muốn đăng ký thực tập với công ty ngoài (không nằm trong danh sách đối tác của khoa).

\begin{table}[H]
	\caption{Đăng ký thực tập với công ty ngoài - Luồng chính}
	\label{tab:register_other_company}
	\begin{tabularx}{\textwidth}{|X|X|X|}
		\hline
		\textbf{Hành động}                                                         & \textbf{Hệ thống phản hồi}                                                           & \textbf{Dữ liệu} \\ \hline
		1. Người dùng truy cập vào trang chủ của hệ thống.                         & 2. Hệ thống hiển thị giao diện trang chủ có phần tìm kiếm tên công ty đang thực tập. &                  \\ \hline
		3. Người dùng nhập tên công ty và chọn công ty có trong danh sách hiện ra. & 4. Hệ thống nhận thông tin và phản hồi lại thông báo.                                & Công ty đã chọn. \\ \hline
	\end{tabularx}
\end{table}

\paragraph*{Luồng thay thế} Không có.

\paragraph*{Luồng ngoại lệ}

\begin{itemize}
	\item

	      Tại bước 3: Người dùng chọn công ty đang ở trạng thái bị từ chối, hệ
	      thống phản hồi thông báo.

	\item

	      Tại bước 4: Hệ thống kiểm định thấy công ty đã chọn đang ở trạng thái
	      bị từ chối, gửi lại thông báo cho người dùng. Người dùng có thể tiếp
	      tục lặp lại luồng sự kiện hoặc thoát khỏi trang hiện tại.

\end{itemize}

\paragraph*{Yêu cầu đặc biệt}

Người dùng là \textbf{sinh viên}.

\paragraph*{Điều kiện đầu}

Người dùng là đã đăng nhập thành công và mong muốn đăng ký thực tập tại
công ty ngoài đã có trong danh sách.

\paragraph*{Điều kiện cuối}

Nếu ca sử dụng thành công, người dùng đã đăng ký thực tập thành công tại
công ty đã chọn, ngược lại, trạng thái hệ thống không thay đổi.

\paragraph*{Các vấn đề mở}

Không có.

\subsection{Đăng ký phỏng vấn từ bài đăng}

\paragraph*{Mô tả tóm tắt}

Người dùng chưa có nơi thực tập, sau khi đọc bài đăng tuyển dụng trên hệ
thống, mong muốn đăng ký phỏng vấn thực tập tại các công ty đã đăng bài.

\paragraph*{Luồng chính} Ca sử dụng bắt đầu khi người dùng muốn đăng ký phỏng vấn với một công ty trong kỳ thực tập.

\begin{table}[H]
	\caption{Đăng ký phỏng vấn từ bài đăng - Luồng chính}
	\label{tab:apply_post}
	\begin{tabularx}{\textwidth}{|X|X|X|}
		\hline
		\textbf{Hành động}                                                      & \textbf{Hệ thống phản hồi}                            & \textbf{Dữ liệu}                         \\ \hline
		1. Người dùng truy cập trang chủ hệ thống.                              & 2. Hệ thống hiển thị danh sách bài đăng tuyển dụng.   &                                          \\ \hline
		3. Người dùng sau khi xem chi tiết bài đăng, yêu cầu đăng ký phỏng vấn. & 4. Hệ thống nhận yêu cầu và gửi thông báo thành công. & Bài đăng tuyển dụng của công ty đã chọn. \\ \hline
	\end{tabularx}
\end{table}

\paragraph*{Luồng thay thế}

\begin{itemize}
	\item

	      Tại bước 3: Người dùng không xem chi tiết bài đăng, mà yêu cầu đăng ký
	      ngay trực tiếp ở trang chủ danh sách bài đăng.

\end{itemize}

\paragraph*{Luồng ngoại lệ} Không có.

\paragraph*{Yêu cầu đặc biệt}

Người dùng là \textbf{sinh viên}.

\paragraph*{Điều kiện đầu}

Người dùng đã đăng nhập thành công vào hệ thống và mong muốn đăng ký
phỏng vấn thực tập sau khi đọc bài đăng tuyển dụng.

\paragraph*{Điều kiện cuối}

Nếu ca sử dụng thành công, người dùng đã đăng ký phỏng vấn thành công
tại công ty đã chọn, ngược lại, trạng thái hệ thống không thay đổi.

\paragraph*{Các vấn đề mở}

Không có.

\subsection{Xem thông tin thực tập}

\paragraph*{Mô tả tóm tắt}

Người dùng mong muốn xem thông tin thực tập của bản thân.

\paragraph*{Luồng chính} Ca sử dụng bắt đầu khi người dùng đã đăng nhập và muốn xem thông tin thực tập trong các kỳ của mình.

\begin{table}[H]
	\caption{Xem thông tin thực tập - Luồng chính}
	\label{tab:read_internship_info}
	\begin{tabularx}{\textwidth}{|X|X|X|}
		\hline
		\textbf{Hành động}                                         & \textbf{Hệ thống phản hồi}                                                     & \textbf{Dữ liệu} \\ \hline
		1. Người dùng yêu cầu xem thông tin thực tập của bản thân. & 2. Hệ thống chuyển hướng hiển thị giao diện thông tin thực tập của người dùng. &                  \\ \hline
	\end{tabularx}
\end{table}

\paragraph*{Luồng thay thế} Không có.

\paragraph*{Luồng ngoại lệ} Không có.

\paragraph*{Yêu cầu đặc biệt}

Người dùng là \textbf{sinh viên}.

\paragraph*{Điều kiện đầu}

Người dùng đã đăng nhập vào hệ thống và mong muốn xem thông tin thực tập
của bản thân.

\paragraph*{Điều kiện cuối}

Nếu ca sử dụng thành công, người dùng đã có thể xem thông tin thực tập
của bản thân, ngược lại, trạng thái hệ thống không thay đổi.

\paragraph*{Các vấn đề mở}

Không có.

\subsection{Chọn công ty để thực tập}

\paragraph*{Mô tả tóm tắt}

Người dùng đã đăng ký thực tập thành công hoặc phỏng vấn đạt thì sau khi
kết thúc thời gian đăng ký, người dùng chọn một công ty duy nhất mà mình
đang và sẽ thực tập tại đó.


\paragraph*{Luồng chính} Ca sử dụng bắt đầu khi người dùng đã đỗ phỏng vấn của một hoặc nhiều công ty và muốn chọn một công ty duy nhất để thực tập.

\begin{table}[H]
	\caption{Chọn công ty để thực tập - Luồng chính}
	\label{tab:select_company}
	\begin{tabularx}{\textwidth}{|X|X|X|}
		\hline
		\textbf{Hành động}                                                                                                           &
		\textbf{Hệ thống phản hồi}                                                                                                   &
		\textbf{Dữ liệu}                                                                                                               \\ \hline
		1. Người dùng truy cập trang thông tin thực tập của bản thân.                                                                &
		2. Hệ thống chuyển hướng hiển thị giao diện thông tin thực tập, bao gồm danh sách công ty đã đăng ký và trạng thái.          &
		\\ \hline
		3. Người dùng chọn một công ty duy nhất trong danh sách công ty đang ở trạng thái PASSED để chuyển sang trạng thái SELECTED. &
		4. Hệ thống ghi nhận yêu cầu, phản hồi thông báo và cập nhật trạng thái thực tập.                                            &
		Công ty đã chọn.                                                                                                               \\ \hline
	\end{tabularx}
\end{table}

\paragraph*{Luồng thay thế} Không có.

\paragraph*{Luồng ngoại lệ} Không có.

\paragraph*{Yêu cầu đặc biệt}

Người dùng là \textbf{sinh viên} và đã có ít nhất một công ty đang ở
trạng thái \emph{PASSED}.

\paragraph*{Điều kiện đầu}

Người dùng đã đăng nhập, đã truy cập vào trang thông tin thực tập và
mong muốn chọn một công ty cuối cùng để thực tập tại đó.

\paragraph*{Điều kiện cuối}

Nếu ca sử dụng thành công, trạng thái thực tập của người dùng đã được
thay đổi và hiển thị ngay trên trang thông tin thực tập.

\paragraph*{Các vấn đề mở}

Không có.

\subsection{Nộp báo cáo thực tập}

\paragraph*{Mô tả tóm tắt}

Sau khi hết thời gian thực tập, người dùng viết báo cáo rồi nộp tại trang Thông tin thực tập trong thời gian cho phép.

\paragraph*{Luồng chính} Ca sử dụng bắt đầu khi người dùng muốn nộp báo cáo thực tập.

\begin{table}[H]
	\caption{Nộp báo cáo thực tập - Luồng chính}
	\label{tab:upload_report}
	\begin{tabularx}{\textwidth}{|X|X|X|}
		\hline
		\textbf{Hành động}                                                                        &
		\textbf{Hệ thống phản hồi}                                                                &
		\textbf{Dữ liệu}                                                                            \\ \hline
		1. Người dùng truy cập trang thông tin thực tập của bản thân.                             &
		2. Hệ thống chuyển hướng hiển thị giao diện thông tin thực tập, bao gồm phần nộp báo cáo. &
		\\ \hline
		3. Người dùng thực hiện nộp báo cáo.                                                      &
		4. Hệ thống ghi nhận thông tin và phản hồi thông báo.                                     &
		File báo cáo.                                                                               \\ \hline
	\end{tabularx}
\end{table}

\paragraph*{Luồng thay thế} Không có.

\paragraph*{Luồng ngoại lệ} Không có.

\paragraph*{Yêu cầu đặc biệt}

Người dùng là \textbf{sinh viên} và đang trong thời gian nộp báo cáo.

\paragraph*{Điều kiện đầu}

Người dùng đã đăng nhập thành công, đã truy cập vào trang thông tin thực tập và mong muốn nộp báo cáo.

\paragraph*{Điều kiện cuối}

Nếu ca sử dụng thành công, người dùng đã nộp báo cáo thành công.

\paragraph*{Các vấn đề mở}

Không có.

\subsection{Tải lên CV}

\paragraph*{Mô tả tóm tắt}

Người dùng mong muốn tải lên CV của bản thân.

\paragraph*{Luồng chính} Ca sử dụng bắt đầu khi người dùng muốn tải lên CV của bản thân.

\begin{table}[H]
	\caption{Tải lên CV - Luồng chính}
	\label{tab:upload_cv}
	\begin{tabularx}{\textwidth}{|X|X|X|}
		\hline
		\textbf{Hành động}                              & \textbf{Hệ thống phản hồi}                                                              & \textbf{Dữ liệu} \\ \hline
		1. Người dùng truy cập trang thông tin cá nhân. & 2. Hệ thống chuyển hướng hiển thị giao diện thông tin cá nhân, bao gồm phần tải lên CV. &                  \\ \hline
		3. Người dùng thực hiện tải lên CV.             & 4. Hệ thống ghi nhận thông tin và phản hồi thông báo.                                   & File CV.         \\ \hline
	\end{tabularx}
\end{table}

\paragraph*{Luồng thay thế} Không có.

\paragraph*{Luồng ngoại lệ} Không có.

\paragraph*{Yêu cầu đặc biệt}

Người dùng là \textbf{sinh viên}.

\paragraph*{Điều kiện đầu}

Người dùng đã đăng nhập thành công, đã truy cập vào trang thông tin cá nhân và mong muốn tải lên CV.

\paragraph*{Điều kiện cuối}

Nếu ca sử dụng thành công, người dùng đã tải lên CV thành công.

\paragraph*{Các vấn đề mở}

Không có.

\subsection{Xem thống kê dữ liệu trên hệ thống}

\paragraph*{Mô tả tóm tắt}

Người dùng xem thống kê dữ liệu theo Role của bản thân.

\paragraph*{Luồng chính} Ca sử dụng bắt đầu khi người dùng muốn xem thống kê dữ liệu.

\begin{table}[H]
	\caption{Xem thống kê dữ liệu - Luồng chính}
	\label{tab:view_dashboard}
	\begin{tabularx}{\textwidth}{|X|X|X|}
		\hline
		\textbf{Hành động}                                  & \textbf{Hệ thống phản hồi}                                                                              & \textbf{Dữ liệu}             \\ \hline
		1. Người dùng truy cập vào trang chủ hệ thống.      & 2. Hệ thống chuyển hướng hiển thị giao diện thống kê dữ liệu trên trang.                                &                              \\ \hline
		3. Người dùng nhập vào mật khẩu cũ và mật khẩu mới. & 4. Hệ thống kiểm định lại mật khẩu cũ và sự phù hợp của mật khẩu mới, rồi thông báo lại cho người dùng. & Mật khẩu cũ và mật khẩu mới. \\ \hline
	\end{tabularx}
\end{table}

\paragraph*{Luồng thay thế} Không có.

\paragraph*{Luồng ngoại lệ} Không có.

\paragraph*{Yêu cầu đặc biệt}

Người dùng là \textbf{giảng viên, đối tác, quản trị viên Khoa, quản trị viên.}.

\paragraph*{Điều kiện đầu}

Người dùng đã đăng nhập vào hệ thống và truy cập trang chủ hệ thống.

\paragraph*{Điều kiện cuối}

Nếu ca sử dụng thành công, người dùng đã ở trang hiển thị thống kê dữ liệu.

\paragraph*{Các vấn đề mở}

Không có.

\subsection{Xem danh sách sinh viên đang hướng dẫn}

\paragraph*{Mô tả tóm tắt}

Người dùng mong muốn xem danh sách sinh viên đang hướng dẫn.

\paragraph*{Luồng chính} Ca sử dụng bắt đầu khi người dùng mong muốn xem danh sách sinh viên đang hướng dẫn.

\begin{table}[H]
	\caption{Xem danh sách sinh viên đang hướng dẫn - Luồng chính}
	\label{tab:view_assigned_students}
	\begin{tabularx}{\textwidth}{|X|X|X|}
		\hline
		\textbf{Hành động}                                               & \textbf{Hệ thống phản hồi}                                                                        & \textbf{Dữ liệu} \\ \hline
		1. Người dùng truy cập trang Danh sách sinh viên đang hướng dẫn. & 2. Hệ thống chuyển hướng hiển thị giao diện Danh sách các sinh viên mà người dùng đang hướng dẫn. &                  \\ \hline
	\end{tabularx}
\end{table}

\paragraph*{Luồng thay thế} Không có.

\paragraph*{Luồng ngoại lệ} Không có.

\paragraph*{Yêu cầu đặc biệt}

Người dùng là \textbf{giảng viên}.

\paragraph*{Điều kiện đầu}

Người dùng đã đăng nhập thành công và truy cập vào trang Danh sách sinh viên đang hướng dẫn.

\paragraph*{Điều kiện cuối}

Nếu ca sử dụng thành công, người dùng đã ở trang hiển thị Danh sách sinh viên đang hướng dẫn.

\paragraph*{Các vấn đề mở}

Không có.

\subsection{Xem báo cáo thực tập của sinh viên}

\paragraph*{Mô tả tóm tắt}

Người dùng mong muốn xem báo cáo của một sinh viên.

\paragraph*{Luồng chính} Ca sử dụng bắt đầu khi người dùng mong muốn xem báo cáo của một sinh viên.

\begin{table}[H]
	\caption{Xem báo cáo thực tập của sinh viên - Luồng chính}
	\label{tab:view_report}
	\begin{tabularx}{\textwidth}{|X|X|X|}
		\hline
		\textbf{Hành động}                                     & \textbf{Hệ thống phản hồi}                                                    & \textbf{Dữ liệu} \\ \hline
		1. Người dùng yêu cầu tải xuống báo cáo của sinh viên. & 2. Hệ thống ghi nhận yêu cầu và trả về cho người dùng file báo cáo tương ứng. &                  \\ \hline
	\end{tabularx}
\end{table}

\paragraph*{Luồng thay thế} Không có.

\paragraph*{Luồng ngoại lệ} Không có.

\paragraph*{Yêu cầu đặc biệt}

Người dùng là \textbf{giảng viên} và sinh viên đó đã nộp báo cáo.

\paragraph*{Điều kiện đầu}

Người dùng đã đăng nhập thành công và mong muốn xem báo cáo của một sinh viên.

\paragraph*{Điều kiện cuối}

Nếu ca sử dụng thành công, người dùng đã tải xuống thành công file báo cáo của sinh viên.

\paragraph*{Các vấn đề mở}

Không có.

\subsection{Chấm điểm cho sinh viên}

\paragraph*{Mô tả tóm tắt}

Người dùng thực hiện chấm điểm cho sinh viên.

\paragraph*{Luồng chính} Ca sử dụng bắt đầu khi người dùng mong muốn chấm điểm cho sinh viên.

\begin{table}[H]
	\caption{Chấm điểm cho sinh viên - Luồng chính}
	\label{tab:score}
	\begin{tabularx}{\textwidth}{|X|X|X|}
		\hline
		\textbf{Hành động}                             & \textbf{Hệ thống phản hồi}                                                        & \textbf{Dữ liệu} \\ \hline
		1. Người dùng yêu cầu Thêm điểm cho sinh viên. & 2. Hệ thống ghi nhận yêu cầu và hiển thị phần nhập điểm cho sinh viên.            &                  \\ \hline
		3. Người dùng nhập điểm và thực hiện gửi lên.  & 4. Hệ thống ghi nhận thông tin gửi lên, cập nhật thông tin và phản hồi thông báo. & Điểm số.         \\ \hline
	\end{tabularx}
\end{table}

\paragraph*{Luồng thay thế}

\begin{itemize}
	\item

	      Tại bước 1: Người dùng yêu cầu nhập điểm bằng cách tải file điểm lên.

	\item

	      Tại bước 2:  Hệ thống hiển thị phần tải lên file điểm.

	\item

	      Tại bước 3: Người dùng tải lên file điểm.

	\item

	      Tại bước 4: Hệ thống ghi nhận file điểm gửi lên, cập nhật thông tin điểm theo file và phản hồi thông báo.

\end{itemize}

\paragraph*{Luồng ngoại lệ} Không có.

\paragraph*{Yêu cầu đặc biệt}

Người dùng là \textbf{giảng viên}, đang hướng dẫn ít nhất là một sinh viên và những sinh viên đó đã nộp báo cáo.

\paragraph*{Điều kiện đầu}

Người dùng đã đăng nhập thành công và mong muốn thực hiện chấm điểm cho sinh viên.

\paragraph*{Điều kiện cuối}

Nếu ca sử dụng thành công, người dùng đã chấm điểm thành công cho sinh viên.

\paragraph*{Các vấn đề mở}

Không có.

% CUT

\subsection{Các ca sử dụng khác}

Ngoài các ca sử dụng đặc tả ở trên, hệ thống còn có những ca sử dụng sau. Các ca sử dụng này sẽ được mô tả cụ thể trong tài liệu của bạn Phạm Thị Dân.

\begin{itemize}
	\item Quản lý bài đăng (cho đối tác)
	\item Quản lý yêu cầu thực tập của sinh viên
	\item Quản lý sinh viên đang thực tập tại công ty
	\item Quản lý kỳ thực tập
	\item Quản lý danh sách sinh viên đang thực tập
	\item Quản lý danh sách đối tác trong kỳ thực tập
	\item Quản lý bài đăng của các đối tác trong kỳ thực tập
	\item Quản lý giảng viên hướng dẫn trong kỳ thực tập
	\item Quản lý danh sách sinh viên
	\item Quản lý danh sách đối tác
	\item Quản lý danh sách giảng viên
	\item Quản lý danh sách người dùng
	\item Quản lý danh sách khoa
	\item Quản lý danh sách lớp
\end{itemize}

\end{document}